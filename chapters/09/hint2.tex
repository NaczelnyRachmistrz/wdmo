\hints{Indukcja matematyczna 2}

\begin{hints_list}
	\item Wykaż, że kwoty $8$, $9$ i $10$ złotych da się opłacić bez wydawania reszty.
	\item Gdy wyróżnione punkty są przykryte prostą, to można zaaplikować założenie indukcyjne.
	\item Szukana liczba będzie miała $1000$ cyfr.
	\item Usuń wierzchołek, który ma co najwyżej $k$ znajomych. Najpierw jednak wykaż, że takowy istnieje.
	\item Zauważ, że jeśli $f(n) \leqslant n - 1$, to z założenia indukcyjnego można stwierdzić, że $f(f(n)) = f(n)$.
	\item Rozpatrz taką podrodzinę rodziny $\mathcal{R}$, że żadne dwa jej elementy nie mają niepustego przecięcia oraz jest to największa możliwa taka podrodzina.
\end{hints_list}