\hints{Indukcja matematyczna 2}

\begin{hints_list}
	\item Zauważ, że jeśli kwotę $k$ złotych da się opłacić w szukany sposób, to kwotę $k + 3$ również.
	\item Rozpatrz punkty $(a, b)$, dla których zachodzi równość $a + b = n + 1$.
	\item Spróbuj rozwiązać zadanie dla liczb podzielnych przez $2$, $4$, i $8$.
	\item Usuń pewien specyficzny wierzchołek i zastosuj założenie indukcyjne.
	\item Wstaw $n = 1$. Wykaż, że $f(n) = n$ korzystając z indukcji zupełnej po $n$.
	\item Wykaż ogólniejszą tezę:
	Niech $\mathcal{R}$ będzie rodziną zbiorów $n$-elementowych. Moc $\mathcal{R}$ jest większa niż $n! \cdot k^{n}$ Wówczas istnieje $k+1$-elementowa rodzina $\mathcal{G}$, będąca podrodziną $\mathcal{R}$, oraz taki zbiór $X$, że dla dowolnych zbiorów $A, B \in \mathcal{R}$ zachodzi $A \cap B = X$.
\end{hints_list}