
\begin{problem}{1}
	Udowodnić, że za pomocą monet trzyzłotowych i pięciozłotowych można zapłacić każdą kwotę większą niż $8$ złotych, bez konieczności wydawania reszty.
\end{problem}

\begin{problem}{2}
	Dana jest liczba naturalna $n$. Niech $\mathcal{S}$ będzie zbiorem wszystkich punktów postaci $(a, b)$, gdzie $a$ i $b$ są liczbami ze zbioru $\{0, 1, 2, ..., n\}$ oraz $a + b \leqslant n$. 

	\begin{center}
	\begin{tikzpicture}[scale = 0.5]
	    \tkzDefPoint(0,0){v_1}
	    \tkzDefPoint(1,0){v_2}
	    \tkzDefPoint(2,0){v_3}
	    \tkzDefPoint(3,0){v_4}
	    \tkzDefPoint(0,1){v_5}
	    \tkzDefPoint(1,1){v_6}
	    \tkzDefPoint(2,1){v_7}
	    \tkzDefPoint(0,2){v_8}
	    \tkzDefPoint(1,2){v_9}
	    \tkzDefPoint(0,3){v_{10}}
	    \tkzDrawPoints(v_1,v_2, v_3, v_4, v_5, v_6, v_7, v_8, v_9, v_{10})
	\end{tikzpicture}\\
	\vspace{5px}
	Przykład dla $n = 3$.
	\end{center}

	\noindent
	Wykazać, że jeśli pewien zbiór prostych zawiera każdy z tych punktów, to zawiera on co najmniej $n + 1$ prostych. 
\end{problem}

\begin{problem}{3}
	Wykazać, że istnieje taka dodatnia liczba całkowita, że jest ona podzielna przez $2^{1000}$, oraz ma w zapisie dziesiętnym jedynie cyfry $1$ i $2$.
\end{problem}

\begin{problem}{4}
	Niech $k$ będzie dodatnią liczbą całkowitą. Na przyjęciu spotkało sie $n \geqslant 2$ gości, spośród których niektórzy znają się. Okazało się, że dla każdego niepustego podzbioru gości $A$ istnieje osoba, która zna co najwyżej $k$ osób z $A$. Podzbiór gości, spośród których każde dwie się znają, nazywamy \textit{kilką}. Wykazać, że istnieje co najwyżej $2^k \cdot n$ klik.
\end{problem}

\begin{problem}{5}
	Znajdź wszystkie funkcje $f$ z dodatnich liczb całkowitych w dodatnie liczby całkowite, które dla każdej liczby dodatniej całkowitej $n$ spełniają nierówności
	\[
		(n - 1)^2 < f(n)f(f(n)) < n^2 + n.
	\]
\end{problem}

\begin{problem}{6}
	Niech $\mathcal{R}$ będzie rodziną zbiorów $1000$-elementowych. Moc $\mathcal{R}$ jest większa niż $1000 \cdot 999^{1000}$ Wykazać, że istnieje $1000$-elementowa rodzina $\mathcal{G}$, będąca podrodziną $\mathcal{R}$, oraz taki zbiór $X$, że dla dowolnych zbiorów $A, B \in \mathcal{R}$ zachodzi $A \cap B = X$.
\end{problem}

