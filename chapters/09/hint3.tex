\hints{Indukcja matematyczna 2}

\begin{hints_list}
	\item *
	\item Jeśli wyróżnione punkty nie są przykryte przez jedną prostą, to do ich przykrycia trzeba co najmniej $n + 1$ prostych.
	\item Powiemy, że liczba naturalna $x$ jest \textit{dobra}, jeśli ma dokładnie $n$ cyfr, z których każda jest jedynką lub dwójką, oraz $x$ jest podzielna przez liczbę $2^n$. Wykaż, że dla dowolnej dodatniej liczby całkowitej istnieje liczba dobra, która ma $n$ cyfr.
	\item Zauważ, że klika zawierająca wierzchołek $X$ składa się z wierzchołka $X$ oraz pewnego podzbioru zbioru jego znajomych.
	\item Gdy $f(n) \geqslant n + 1$, to $f(f(n)) \leqslant n - 1$. Wstaw $f(n)$ zamiast $n$ do wyjściowej nierówności.
	\item Wykaż, że pewien element należy do co najmniej $(n - 1)! \cdot k^{n - 1}$ zbiorów.
\end{hints_list}