\newpage
\solutions{Wielomiany}

\begin{problem}{1}
    Dany jest niezerowy wielomian $W(x)$ o współczynnikach rzeczywistych, dla którego zachodzi równość
    \[
        (x + 1)W(x) = (x - 2)W(x + 1).
    \]
    Wykazać, że każdy pierwiastek rzeczywisty $W(x)$ jest liczbą całkowitą.
\end{problem}

\noindent
Załóżmy nie wprost, że istnieje pewna liczba niecałkowita $\alpha$, dla której zachodzi równość $W(\alpha) = 0$. Wstawiając $x = \alpha$ do wyjściowego równania otrzymujemy
\begin{align*}
    (\alpha + 1)W(\alpha) = (\alpha - 2)W(\alpha + 1), \\
    0 = (\alpha - 2)W(\alpha + 1).
\end{align*}
Skoro liczba $\alpha$ nie była całkowita, to $\alpha - 2 \neq 0$. Stąd $W(\alpha + 1) = 0$.

\vspace{10 px}
\noindent
Wykazaliśmy, że jeśli $\alpha$ jest niecałkowitym pierwiastkiem, to również $\alpha + 1$ jest niecałkowitym pierwiastkiem. Więc liczby
\[
    \alpha, \; \alpha + 1, \; \alpha + 2, \; \alpha + 3, \; ...
\]
są pierwiastkami $W(x)$. Zaś każdy niezerowy wielomian może mieć skończenie wiele pierwiastków -- co najwyżej tyle, ile wynosi jego stopień -- co daje sprzeczność.


\begin{problem}{2}
    Wykazać, że jeśli liczby rzeczywiste $a, b, c$ spełniają
    \[
    \begin{cases}
        abc = 1 \\
        \frac{1}{a} + \frac{1}{b} + \frac{1}{c} = a + b + c.
    \end{cases}
    \]
    to co najmniej jedna liczba spośród $a, b, c$ jest równa 1.
\end{problem}

\noindent
Rozpatrzmy wielomian
\begin{align*}
    P(x) &= (x - a)(x - b)(x - c) = x^3 - (a + b + c)x^2 + (ab + bc + ca)x - abc = \\
    &= x^3 - (a + b + c)x^2 + abc \cdot \left(\frac{1}{a} + \frac{1}{b} + \frac{1}{c}\right)x - 1 = \\
    &= x^3 - (a + b + c)x^2 + (a + b + c)x - 1.
\end{align*}
Chcemy wykazać, że liczba $1$ jest pierwiastkiem tego wielomianu. Zauważmy więc, że
\[
    P(1) = 1^3 - (a + b + c) \cdot 1^2 + (a + b + c)\cdot 1 - 1 = 0.
\]
Skoro $1$ jest pierwiastkiem $P(x)$, to należy on do multizbioru $\{a, b, c\}$.

\begin{problem}{3}
    Wykazać, że wielomian $x^{1001} + x + 1$ jest podzielny przez $x^2 + x + 1$.
\end{problem}

\noindent
Wykażemy indukcyjnie, że dla każdej dodatniej liczby całkowitej $n$ wielomian $x^{3n - 1} + x + 1$ jest podzielny przez wielomian $x^2 + x + 1$. Dla $n = 1$ te wielomiany są sobie równe, więc podzielność zachodzi.
 
\vspace{10 px}
\noindent
Zauważmy, że
\[
    (x^{3(n + 1) - 1} + x + 1) - (x^{3n - 1} + x + 1) = x^{3n - 1}(x^3 - 1) =   x^{3n - 1}(x - 1)(x^2 + x + 1).
\]
Prawa strona równości jest podzielna przez wielomian $x^2 + x + 1$. Stąd jeśli $x^{3n - 1} + x + 1$ jest podzielne przez $x^2 + x + 1$ to $x^{3(n + 1) - 1} + x + 1$ również. Więc z zasady indukcji matematycznej wynika postulowana własność. Wstawiając $n = 334$ otrzymujemy tezę.


\begin{problem}{4}
    Dany jest pewien wielomian $P(x)$. Wykazać, że istnieje wielomian $Q(x)$, że zachodzi równość
    \[
        Q(x + 1) - Q(x) = P(x).
    \]
\end{problem}

\noindent
Tezę wykażemy indukując się po stopniu wielomianu $P$. Jeśli jego stopień wynosi $0$, to znaczy $P(x) = c$ dla pewnej stałej $c$, to biorąc $Q(x) = cx$ otrzymujemy
\[
    Q(x + 1) - Q(x) = c(x + 1) - cx = c.
\]
Załóżmy, że dla wielomianów o stopniu nie większym niż $n - 1$ zachodzi teza. Wykażemy, że jeśli ${deg\; P = n}$ to szukane $Q$ istnieje.
Zauważmy, że wielomian 
\[
    (x + 1)^{n + 1} - x^{n + 1} = \sum^{n}_{i = 0} {{n + 1}\choose{i}}x^i
\]
jest wielomianem stopnia $n$. Taki sam stopień ma $P$, więc istnieje taka liczba rzeczywista~$a$, że
\[
    a\left((x + 1)^{n + 1} - x^{n + 1} \right) \quad \text{oraz} \quad P(x)
\]
mają równy współczynnik wiodący. Stąd
\[
    P(x) - a\left((x + 1)^{n + 1} - x^{n + 1} \right)
\]
ma stopień mniejszy od $n$, czyli na mocy założenia indukcyjnego istnieje takie $Q_1(x)$, że
\[
    Q_1(x + 1) - Q_1(x) = P(x) - a\left((x + 1)^{n + 1} - x^{n + 1} \right),
\]
równoważnie
\[
    (Q_1(x + 1) + a(x + 1)^{n + 1}) - (Q_1(x) + ax^{n + 1}) = P(x).
\]
Biorąc $Q(x) = Q_1(x) + ax^{n + 1}$ otrzymujemy tezę.

\begin{problem}{5}
    Dany jest wielomian $W(x)$ o współczynnikach całkowitych. Wykazać, że jeśli przyjmuje on dla czterech różnych liczb całkowitych wartość $5$, to dla żadnego całkowitego argumentu nie przyjmuje wartości $8$
\end{problem}

\noindent
Przyjmijmy, że
\[
    W(a_1) = W(a_2) = W(a_3) = W(a_4) = 5 \quad \text{oraz} \quad W(b) = 8.
\]
dla pewnych liczb całkowitych $a_1, \; a_2, \; a_3, \; a_4$.
Rozpatrzmy wielomian
\[
    Q(x) = W(x) - 5.
\]
Wówczas liczby $a_1, \; a_2, \; a_3, \; a_4$ są jego pierwiastkami. Na mocy twierdzenia Bézouta możemy napisać
\[
    Q(x) = (x - a_1)(x - a_2)(x - a_3)(x - a_4) \cdot R(x)
\]
dla pewnego wielomianu $R$. Zauważmy, że $R(x)$ musi mieć współczynniki całkowite. W~przeciwnym wypadku, rozpatrując współczynnik niecałkowity przy najwyższej możliwej potędze, przemnożony przez $x^4$ otrzymalibyśmy, że $Q(x)$ też ma współczynnik niecałkowity.
Zauważmy, że
\[
    3 = W(b) - 5 = Q(b) = (b - a_1)(b - a_2)(b - a_3)(b - a_4) \cdot R(b).
\]
Liczby $b - a_1, \; b - a_2, \;b - a_3, \;b - a_4$ są parami różne. Stąd co najmniej dwie z nich nie są równe $-1$ ani $1$. Stąd wartość bezwzględna ich iloczynu nie może być liczbą pierwszą, w szczególności być równa $3$.



\begin{problem}{6}
    Rozstrzygnąć, czy istnieje wielomian $1000$ stopnia, którego współczynniki należą do zbioru $\{-1, 1\}$ oraz ma on $1000$ pierwiastków rzeczywistych.
\end{problem}

\noindent
Wykażemy, że szukany wielomian nie istnieje. Przyjmijmy, że ma on pierwiastki $a_1$, $a_2$, ..., $a_{1000}$. Wówczas jest on postaci
\[
    W(x) = a(x - a_1)(x - a_2)\cdot ... \cdot (x - a_{1000}).
\]
Wymnażając te nawiasy możemy zauważyć, że wyraz wolny wynosi $a_1a_2 \cdot ... \cdot a_{1000}$, współczynnik przy $x$ będzie równy $\sum_{1 \leqslant i \leqslant 1000} a_i$, zaś współczynnik przy $x^2$ wyniesie $\sum_{1 \leqslant i < j \leqslant 1000} a_ia_j$. Na mocy założeń są one równe $\pm 1$. Mamy
\begin{align*}
    \sum_{1 \leqslant i \leqslant 1000} a_i = \pm 1, \\
    \left(\sum_{1 \leqslant i \leqslant 1000} a_i\right)^2 = 1, \\
   \sum_{1 \leqslant i \leqslant 1000} a_i^2 + 2\sum_{1 \leqslant i < j \leqslant 1000} a_ia_j = 1. \\
\end{align*}
Jeśli $\sum_{1 \leqslant i < j \leqslant 1000} a_ia_j = 1$, to na mocy powyższej równości mielibyśmy, że suma kwadratów jest ujemna. Stąd $\sum_{1 \leqslant i < j \leqslant 1000} a_ia_j = -1$, czyli
\[
    \sum_{1 \leqslant i \leqslant 1000} a_i^2 = 1 - 2\sum_{1 \leqslant i < j \leqslant 1000} a_ia_j = 3.
\]
Na mocy nierówności między średnią arytmetyczną i geometryczną mamy
\[
   \frac{3}{1000} =  \frac{\sum_{1 \leqslant i \leqslant 1000} a_i^2}{1000} \geqslant \sqrt[1000]{|a_1a_2\cdot ... \cdot a_{1000}|^2} = 1,
\]
co daje nam sprzeczność.

