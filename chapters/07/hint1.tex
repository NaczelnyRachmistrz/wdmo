\hints{Wielomiany}

\begin{hints_list}
	\item Załóż nie wprost, że istnieje pewne niecałkowite $\alpha$, że $W(\alpha) = 0$.
	\item Rozpatrz wielomian o pierwiastkach $a$, $b$ i $c$.
	\item Wykaż, że dla każdej dodatniej liczby całkowitej $n$ wielomian $x^{3n - 1} + x + 1$ jest podzielny przez wielomian $x^2 + x + 1$.
	\item Przeprowadź indukcję po stopniu $P$.
	\item Rozpatrz wielomian $Q(x) = W(x) - 5$.
	\item Wykaż, że Tomek jest w stanie, niezależnie od $n$ wygrać w $2$ ruchach.
	\item Taki wielomian nie istnieje. Skorzystaj z podobnego pomysłu co we wzorach Viete'a we wstepie teoretycznym.
\end{hints_list}