\begin{problem}{1}
	Dany jest niezerowy wielomian $W(x)$ o współczynnikach rzeczywistych, dla którego zachodzi równość
	\[
		(x + 1)W(x) = (x - 2)W(x + 1).
	\]
	Wykazać, że każdy pierwiastek rzeczywisty $W(x)$ jest liczbą całkowitą.
\end{problem}


\begin{problem}{2}
	Wykazać, że jeśli liczby rzeczywiste $a, b, c$ spełniają
    \[
    \begin{cases}
        abc = 1 \\
        \frac{1}{a} + \frac{1}{b} + \frac{1}{c} = a + b + c.
    \end{cases}
    \]
    to co najmniej jedna liczba spośród $a, b, c$ jest równa 1.
\end{problem}

\begin{problem}{3}
	Niech $n \geqslant 1$ będzie pewną liczbą całkowitą. Wykazać, że wielomian $x^{1001} + x + 1$ jest podzielny przez $x^2 + x + 1$.
\end{problem}


\begin{problem}{4}
	Dany jest pewien wielomian $P(x)$. Wykazać, że istnieje wielomian $Q(x)$, że zachodzi równość
	\[
		Q(x + 1) - Q(x) = P(x).
	\]
\end{problem}

\begin{problem}{5}
	Dany jest wielomian $W(x)$ o współczynnikach całkowitych. Wykazać, że jeśli przyjmuje on dla czterech różnych liczb całkowitych wartość $5$, to dla żadnego całkowitego argumentu nie przyjmuje wartości $8$
\end{problem}

\begin{problem}{7}
	Rozstrzygnąć, czy istnieje wielomian $1000$ stopnia, którego współczynniki należą do zbioru $\{-1, 1\}$ oraz ma on $1000$ pierwiastków rzeczywistych.
\end{problem}

