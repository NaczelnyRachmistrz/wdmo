% https://artofproblemsolving.com/community/c6h2133867p15618318
\begin{problem}{1} 
	Wykazać, że można pokolorować $40$ pól na nieskończonej szachownicy, tak, aby nie istniał prostokąt utworzony z pól tej szachownicy zawierający dokładnie $20$ pokolorowanych pól.
\end{problem}

%Source: https://omj.edu.pl/uploads/attachments/ObozOMJ2018.pdf P11
\begin{problem}{2} 
	Udowodnij, że punkty płaszczyzny można tak pokolorować dziewięcioma kolorami, aby żadne dwa punkty odległe o 1 nie były tego samego koloru.
\end{problem}

%Source: https://artofproblemsolving.com/community/c6h2097901p15170095
\begin{problem}{3}
	Wykazać, że każdy trójkąt można podzielić na $3000$ czworokątów wypukłych, tak, aby każdy z nich dało się wpisać w okrąg oraz opisać na okręgu.
\end{problem}

%Source: https://artofproblemsolving.com/community/c6h277245p1500076 Russia 2009 P 9.6
\begin{problem}{4} 
	Wykazać, że można pokolorować każdą dodatnią liczbę całkowitą na jeden z $1000$ kolorów, tak aby
	\begin{itemize}
		\item każdy z kolorów był użyty nieskończenie wiele razy;
		\item dla dowolnych takich liczb całkowitych $a$, $b$, $c$, że $ab = c$, pewne dwie spośród nich sa jednakowego koloru.
	\end{itemize} 
\end{problem}

%Source: https://omj.edu.pl/uploads/attachments/ObozOM2018.pdf P11
\begin{problem}{5}
	Łódka może zabrać w rejs po jeziorze dokładnie 7 osób. Udowodnij. że można tak zaplanować rejsy 49-osobowej wycieczki, aby każdych dwóch uczestników płynęło ze sobą dokładnie raz.
\end{problem}

%Source: https://artofproblemsolving.com/community/c6h589865p3493114 Russia 2014 9.3
\begin{problem}{6}
	Jaś zapisał pewną skończoną liczbę liczb rzeczywistych na tablicy. Następnie zaczął wykonywać ruchy. W każdym ruchu wybiera dwie równe liczby $a$, $a$, zmazuje je i zapisuje liczby $a + 100$, $a + 2020$. Rozstrzygnąć, czy Jaś może zapisać na początku takie liczby, że będzie mógł wykonywać ruchy w nieskończoność.
\end{problem}

