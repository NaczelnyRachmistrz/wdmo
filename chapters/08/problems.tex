\begin{problem}{1}
	W pewnym grafie każdy wierzchołek ma stopień co najmniej $100$. Wykazać, że w tym grafie istnieje ścieżka o długości co najmniej $101$.
\end{problem}

\begin{problem}{2}
	W pewnym kraju jest $n$ miast, przy czym każde dwa są połączone drogą albo torami kolejowymi. Pewien turysta planuje wyruszyć z pewnego miasta, odwiedzić każde miasto dokładnie raz, a następnie powrócić do wyjściowego miasta. Wykazać, że może tak wybrać wyjściowe miasto i tak zaplanować swoją trasę, aby zmienić środek transportu co najwyżej raz.
\end{problem}

%Source: Tournament of the towns 1986
\begin{problem}{3}
	W pewnym turnieju bierze udział $40$ drużyn. Pierwszego dnia każda z drużyn rozegrała jeden mecz. Drugiego dnia również. Wykazać, że istnieje pewne $20$ drużyn, takich, że każde dwie spośród nich jeszcze nie grały ze sobą meczu.
\end{problem}

%Source: Tournament of the towns 1986
\begin{problem}{4}
	W pewnym turnieju bierze udział $40$ drużyn. Pierwszego dnia każda z drużyn rozegrała jeden mecz. Drugiego dnia również. Wykazać, że istnieje pewne $20$ drużyn, takich, że każde dwie spośród nich jeszcze nie grały ze sobą meczu.
\end{problem}


\begin{problem}{5}
	W pewnym grafie o $n$ wierzchołkach ich stopnie wynoszą odpowiednio $d_1$, $d_2$, ..., $d_n$. Udowodnić, że istnieje taki podzbiór co najmniej $\sum^{n}_{i = 1} \frac{1}{1 + d_i}$ jego wierzchołków, że żadne dwa z nich nie są połączone krawędzią.
\end{problem}

%Source: https://artofproblemsolving.com/community/c6h455746p2560781
\begin{problem}{6}
	Hydra składa się z pewnej liczby głów, z której niektóre są połączone szyjami. Herkulers może odciąć wszystkie szyje wychodzące z pewnej głowy, jednak wówczas z tamtej głowy wyrastają szyję, którą łączą ją z głowami, z którymi nie była ona wcześnie połączona. Hydra jest pokonana, gdy rozpada się na dwie rozłączne części. Wyznaczyć najmniejsze $N$, że Herkules jest w stanie pokonać dowolną hydrę składającą się ze $100$ szyi.
\end{problem}
