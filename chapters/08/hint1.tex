\hints{Grafy}

\begin{hints_list}
	\item Rozpatrz najdłuższą ścieżkę w grafie i spróbuj ją wydłużyć.
	\item Skorzystaj z indukcji matematycznej.
	\item Rozpatrz graf, w którym wierzchołkami będą drużyny. Połączymy je czerwoną krawędzią, gdy grały mecz pierwszego dnia, zieloną, gdy drugiego.
	\item Wykaż, że mając pewien graf z pewnym kolorowaniem, to wybierając dwa dowolne wierzchołki $u$ oraz $v$ jesteśmy w stanie zmienić parzystość liczby kolorowych krawędzi wychodzących z nich, nie naruszając tej liczby dla innych wierzchołków.
	\item Wybierz jeden wierzchołek, następnie usuń wszystkich jego sąsiadów. Powtarzaj taki krok z otrzymanym grafem, aż wszystkie wierzchołki zostaną usunięte. Wybierając mądrze wierzchołki możemy zrobić to tak, że wybierzemy ich odpowiednio dużo.
	\item Odpowiedzią jest $N = 10$.
\end{hints_list}