\newpage
\solutions{Grafy}

\begin{problem}{1}
	W pewnym grafie każdy wierzchołek ma stopień co najmniej $100$. Wykazać, że w tym grafie istnieje ścieżka o długości co najmniej $101$.
\end{problem}

\noindent
Rozpatrzmy najdłuższą ścieżkę w tym grafie i załóżmy nie wprost, że jest ona długości nie większej niż $100$. Wierzchołek początkowy tej ścieżki $V$ jest połączony z co najmniej $101$ wierzchołkami. Co najwyżej $99$ z nich leży na rozpatrywanej ścieżce -- $100$ wierzchołków z wyłączeniem $V$. Stąd $V$ jest połączony z pewnym wierzchołkiem spoza ścieżki – możemy więc wydłużyć tę ścieżkę o tego sąsiada. Przeczy to maksymalności długości tej ścieżki.

\begin{center}
    \begin{tikzpicture}
    \tkzDefPoint(0,0){v_1}
    \tkzDefPoint(1,0){v_2}
    \tkzDefPoint(2,0.6){v_3}
    \tkzDefPoint(3,1.2){v_4}
    \tkzDefPoint(-0.5,1){v_5}
    \tkzDrawPoints(v_1,v_2, v_3, v_4, v_5)
    \tkzDrawSegments(v_1,v_2 v_3,v_4  v_2,v_3 v_1,v_4 v_1,v_3)
    \tkzDrawSegments[dashed](v_1,v_5)
    \tkzLabelPoint[below](v_1){$V$}
    \tkzLabelPoint[below](v_2){1}
    \tkzLabelPoint[below](v_3){2}
    \tkzLabelPoint[below](v_4){3}
    \tkzLabelPoint[left](v_5){$X$}
    \end{tikzpicture}\\
    Wydłużamy ścieżkę $V$-1-2-3 o $X$.\\
\end{center}


\begin{problem}{2}
	W pewnym kraju jest $n \geqslant{3}$ miast, przy czym każde dwa są połączone drogą albo torami kolejowymi. Pewien turysta planuje wyruszyć z pewnego miasta, odwiedzić każde miasto dokładnie raz, a następnie powrócić do wyjściowego miasta. Wykazać, że może tak wybrać wyjściowe miasto i tak zaplanować swoją trasę, aby zmienić środek transportu co najwyżej raz.
\end{problem}

\noindent
Przekłóżmy zadanie na język teorii grafów. Miasta będa wierzchołkami, połączenie kolejowe niebieską krawędzią, a połączenie drogowe czerwoną.

\vspace{5px}

\noindent
Wykażemy tezę indukcyjnie. Zauważmy, że dla $n = 3$ jest ona trywialna. Gdy wszystkie krawędzie są tego samego koloru, to dowolne przejście spełnia warunki zadania. Gdy tak nie jest, to zaczynamy w wierzchołku, w którym schodzą się dwie krawędzie różnych kolorów i przechodzimy przez wszystkie krawędzie.

\vspace{5px}

\noindent
Wyróżnijmy pewien wierzchołek $v$. Wszystkie inne wierzchołki tworzą graf, który ma $n - 1$ wierzchołków, a więc możemy skorzystać z założenia indukcyjnego. Załóżmy, że krawędzie $d_1d_2$, $d_2d_3$, ..., $d_{k-1}d_1$ tworzą cykl spełniający warunki zadania. Gdy cykl ten jest jednokolorowy, to jeśli wstawimy $v$ między $d_1$ i $d_2$, to tak otrzymany cykl będzie spełniał założenia. 

\vspace{5px}

\noindent
Rozpatrzmy przypadek, w którym istnieje kolor $\mathcal{K}$, że tylko jedna krawędź jest koloru $x$. Załóżmy, bez straty ogólności, że $d_1d_2$ jest koloru $\mathcal{K}$, a inne krawędzie są innego koloru. Wtedy cykl $d_1v$, $vd_2$, $d_2d_3$, ..., $d_{k-1}d_1$ spełnia warunki zadania. 

\vspace{5px}

\begin{center}
    \begin{tikzpicture}
    \tkzDefPoint(0,0){v_1}
    \tkzDefPoint(2,0){v_2}
    \tkzDefPoint(2,1){v_4}
    \tkzDefPoint(3,0.5){v_3}
    \tkzDefPoint(0,1){v_5}
    \tkzDefPoint(-1,0.5){v_6}
    \tkzDefPoint(1,2){v}
    \tkzDrawPoints(v_1,v_2, v_3, v_4, v_5, v_6, v)
    \tkzDrawSegments(v_1,v_2 v_2,v_3 v_1,v_6 v_5,v_6 v,v_5)
    \tkzDrawSegments[dashed](v_4,v_5 v_3,v_4 v,v_4)
    \end{tikzpicture}
\end{center}


\noindent
Teraz przyjrzyjmy się przypadkowi, w którym każdy z kolorów wystepuje co najmniej dwukrotnie na rozpatrywanym cyklu.
Załóżmy więc bez straty ogólności, że $d_1d_2$ i $d_2d_3$ są czerwone, a $d_3d_4$ i $d_4d_5$ są niebieskie. Przyjmijmy, również bez straty ogólności, że $vd_3$ jest czerwona. Wtedy cykl $d_1d_2$, $d_2d_3$, $d_3v$, $vd_4$, $d_4d_5$, ..., $d_{k-2}d_{k-1}$, $d_{k-1}d_1$ spełnia warunki zadania.

\vspace{5px}

%Source: Tournament of the towns 1986
\begin{problem}{3}
	W pewnym turnieju bierze udział $40$ drużyn. Pierwszego dnia każda z drużyn rozegrała jeden mecz. Drugiego dnia również. Wykazać, że istnieje pewne $20$ drużyn, takich, że każde dwie spośród nich jeszcze nie grały ze sobą meczu.
\end{problem}

\noindent
Rozpatrzmy graf, w którym wierzchołkami będą drużyny. Krawedź będzie istnieć wtedy i tylko wtedy, gdy drużyny odpowiadające jej końcą rozegrały ze sobą mecz. Zauważmy, że warunek z zadania jest równoważny faktowi, że stopień każdego wierzchołka jest równy~2. 

Wybierzmy pewien wierzchołek i rozpocznijmy w nim spacer po grafie. Będziemy przechodzić do następnych wierzchołków krawędziami, którymi jeszcze nie szliśmy. Łatwo zauważyć, że musimy w ten sposób otrzymać cykl, w którym każdy wierzchołek odwiedzamy co najwyżej raz. Skoro stopień każdego z wierzchołków wynosi $2$, to żaden z wierzchołków tego cyklu nie ma krawędzi łączącej go z jakimkolwiek wierzchołkiem spoza tego cyklu. Wykonując analogiczne spacery startując z nieodwiedzonych jeszcze wierzchołków - o ile takowe istnieją - otrzymamy, że graf jest sumą rozłącznych cykli.

Pokolorujmy każdą z krawędzi na czerwono, jeśli mecz odbył się pierwszego dnia, lub na zielono, jeśli odbył się drugiego dnia. Zauważmy, że z każdego wierzchołka wychodzi jedna zielona i jedna czerwona krawędź. Stąd w każdym cyklu krawędzie na przemian: zielona, czerwona, zielona, czerwona, ... itd. Stąd każdy z cykli jest parzystej długości. 

\begin{center}
    \begin{tikzpicture}
    \tkzDefPoint(0,0){v_1}
    \tkzDefPoint(2,0){v_2}
    \tkzDefPoint(2,2){v_4}
    \tkzDefPoint(3,1){v_3}
    \tkzDefPoint(0,2){v_5}
    \tkzDefPoint(-1,1){v_6}
    \tkzDrawPoints(v_1,v_2, v_3, v_4, v_5, v_6)
    \tkzDrawSegments(v_1,v_2 v_3,v_4 v_5,v_6)
    \tkzDrawSegments[dashed](v_2,v_3 v_4,v_5 v_1,v_6)

    \tkzDefPoint(4,0){v_1}
    \tkzDefPoint(6,0){v_2}
    \tkzDefPoint(6,2){v_3}
    \tkzDefPoint(4,2){v_4}
    \tkzDrawPoints(v_1,v_2, v_3, v_4)
    \tkzDrawSegments(v_1,v_2 v_3,v_4)
    \tkzDrawSegments[dashed](v_2,v_3 v_4,v_1)
    \end{tikzpicture}
\end{center}

W każdym z cykli długości parzystej możemy wybrać połowę jego wierzchołków, aby żadne dwa z nich nie grały ze sobą meczu. Wybierając takie wierzchołki dla każdego z cykli otrzymamy połowę wierzchołków z całego grafu, które nie są połączone ze sobą. Odpowiadające im $20$ drużyn spełniają warunki zadania.

\vspace{5px}

\newpage

%Source: Tournament of the Towns Senior-O Spring 2011 P5
\begin{problem}{4}
	Dany jest graf spójny zawierający parzystą liczbę wierzchołków. Wykazać, że możemy pokolorować niektóre z jego krawędzi, aby z każdego wierzchołka wychodziła nieparzysta liczba pokolorowanych krawędzi.
\end{problem}

\noindent
Rozpatrzmy dwa wierzchołki $u$ i $v$. Skoro graf jest spójny, to istnieje różnowierzchołkowa ścieżka między tymi wierzchołkami. Zmieńmy stan -- z pokolorowanej na niepokolorowaną i na odwrót --  każdej z krawędzi na tej ścieżce. Zauważmy, że parzystość liczby kolorowych krawędzi wychodzących z każdego wierzchołka wewnątrz ścieżki nie ulegnie zmianie. Natomiast parzystość liczby kolorowych krawędzi wychodzących z $u$ i z $v$ się zmieni. Czyli jesteśmy w stanie zmienić parzystość liczby kolorowych krawędzi wychodzących z dowolnych dwóch wierzchołków. Skoro liczba wierzchołków w grafie jest liczbą parzystą, to możemy podzielić je w dowolny sposób na pary i zastosować wskazany wyżej algorytm dla każdej z nich. W ten sposób uzyskamy szukane kolorowanie.

\vspace{5px}

\begin{problem}{5}
	W pewnym grafie o $n$ wierzchołkach ich stopnie wynoszą odpowiednio $d_1$, $d_2$, ..., $d_n$. Udowodnić, że istnieje taki podzbiór co najmniej $\sum^{n}_{i = 1} \frac{1}{1 + d_i}$ jego wierzchołków, że żadne dwa z nich nie są połączone krawędzią.
\end{problem}

\noindent
Rozpatrzmy wierzchołek o największym stopniu. Wyróżnijmy go i usuńmy go wraz ze wszystkimi sąsiadującymi wierzchołkami. Bez straty ogólności $d_1 \geqslant d_2 \geqslant ... \geqslant d_{d_1 + 1}$ to będą stopnie usuniętych wierzchołków. Wówczas rozpatrywana suma ułamków zmniejszy się o sumę
\[
	\sum^{d_1 + 1}_{i = 0} \frac{1}{d_i + 1} \geqslant \sum^{d_1 + 1}_{i = 0} \frac{1}{d_1 + 1} = 1.
\]
Także po usunięciu tych wierzchołków inne stopnie mogą się zmniejszyć, ale to jeszcze bardziej zmniejszy naszą sumę.

Wykazaliśmy, że usuwając pewien wierzchołek i jego sąsiadów, zmiejszamy rozpatrywaną sumę o co najwyżej $1$. Powtarzając ten algorytm, wykonamy co najmniej $\sum^{n}_{i = 1} \frac{1}{1 + d_i}$ usunięć. Biorąc pod uwagę wyróżnione wierzchołki otrzymamy zbiór spełniający warunki zadania.

\vspace{5px}

%Source: https://artofproblemsolving.com/community/c6h455746p2560781
\begin{problem}{6}
	Hydra składa się z pewnej liczby głów, z której niektóre są połączone szyjami. Herkulers może odciąć wszystkie szyje wychodzące z pewnej głowy, jednak wówczas z tamtej głowy wyrastają szyję, którą łączą ją z głowami, z którymi nie była ona wcześnie połączona. Hydra jest pokonana, gdy rozpada się na dwie rozłączne części. Wyznaczyć najmniejsze $N$, że Herkules jest w stanie pokonać dowolną hydrę składającą się ze $100$ szyi.
\end{problem}

\noindent
Wykażemy, że $N = 10$. Oczywiście rozpatrzamy dany problem w języku teorii grafów. Rozpatrzmy dwa przypadki.

Jeśli istnieje wierzchołek o stopnie nie większym niż $10$, to Herkules może odciąć każdego z jego sąsiadów i wówczas ten wierzchołek nie będzie miał już żadnych sąsiadów. W ten sposób może osiągnąć swój cel.

Załóżmy teraz, że każdy wierzchołek ma stopień co najmniej $11$. Oznaczmy liczbę wierzchołków jako $K$. Wówczas liczba krawędzi wynosi co najmniej $\frac{11K}{2}$. Skoro jest ona równa $100$, to otrzymujemy $K \leqslant 18$. Zauważmy, że po wykonaniu odcięcia wierzchołka o stopniu co najmniej $11$, będzie on miał stopień co najwyżej $18 - 11 - 1 = 6$. Wówczas stosując procedurę analogiczną do poprzedniego prypadku otrzymamy $6 + 1 = 7$ łacznych cięć.


Pozostaje wskazać przykład grafu, którego nie da rozciąć się w mniej niż $10$ ruchach. Zauważmy, że graf $K_{a,b}$ może zostać przekształcony na jeden z grafów $K_{a - 1,b + 1}$ lub $K_{a + 1,b - 1}$. Rozpatrując graf $K_{10,10}$ otrzymujemy szukany przykład.
