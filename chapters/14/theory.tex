% Rozdział 14 – twierdzenia z teorii liczb

\theory{Twierdzenia z teorii liczb}

% skrypt bardziej teoretyczny
% zadania nie zawsze są z nim związane, ale są pouczające

\noindent
Na potrzeby natstepującego lematu i twierdzenia, dla liczby całkowitej $n$ oraz parami względnie pierwszych liczb $m_1$, ..., $m_k$ przyjmiemy
\[
	G(n) = (n \;(\text{mod } m_1), n \;(\text{mod } m_2), ..., n \;(\text{mod } m_k)).
\]


% obserwacja, że G(n) może przyjmować m_1m_2...m_k wartości

\heading{Lemat 1}

\noindent
Dane są parami względnie pierwsze liczby $m_1$, ..., $m_k$ oraz dodatnie liczby całkowite~$x$,~$y$ spełniające $m_1m_2\cdot ... \cdot m_k > x > y > 0$. Wykazać, że wówczas
\[
	G(x) \neq G(y).
\]

\heading{Dowód}



\heading{Chińskie twierdzenie o resztach}

\noindent
Dane są parami względnie pierwsze liczby $m_1$, ..., $m_k$ oraz pewne liczby całkowite $a_1$, ..., $a_k$. Wówczas istnieje taka liczba całkowita $x$, że zachodzą przystawania
\begin{gather*}
	x \equiv a_1 \pmod{m_1}, \\
	x \equiv a_2 \pmod{m_2}, \\
	..., \\
	x \equiv a_k \pmod{m_k}.
\end{gather*}

\heading{Dowód}

% potem komentarz, że to znaczy tyle, że ,,jakie przystawania sobie zażyczymy, takie mamy

\heading{Symbol Legendre'a}

\noindent
Dla pewnej liczby pierwszej $p$ oraz liczby całkowitej $n$ przyjmiemy
\[
	\left(\frac{n}{p}\right) = 
	\begin{cases}
		\phantom{-}1 \text{, gdy istnieje taka liczba całkowita } a \not\equiv 0 \text{, że } n \equiv a^2 \pmod{p}\\
		-1\text{, gdy nie istnieje taka liczba całkowita } a \not\equiv 0 \text{, że } n \equiv a^2 \pmod{p} \\
		\phantom{-}0 \text{, gdy } n \equiv 0 \pmod{p}\\
	\end{cases}
\]

\noindent
W pierwszym przypadku powiemy, że $n$ jest \textit{resztą kwadratową}, a w drugim, że $n$ jest \textit{nieresztą kwadratową}.


\heading{Twierdzenie 2}

\noindent
Dana jest liczba pierwsza $p$ oraz liczby całkowite $a$, $b$. Wówczas zachodzi równość
\[
	\left(\frac{a}{p}\right)\left(\frac{b}{p}\right) = \left(\frac{ab}{p}\right).
\]

\heading{Dowód}

\heading{Pierwiastki pierwotne modulo $p$}

% bez dowodu!!!

\vspace{10px}


\vspace{10px}

\heading{Przykład 1}

\noindent
Wykazać, że dla dodatniej liczby całkowitej, która nie jest podzielna przez $p - 1$ zachodzi przystawanie
\[
	1^k + 2^k + 3^k + ... + p^k \equiv 0 \pmod{p}.
\]

\heading{Rozwiązanie}