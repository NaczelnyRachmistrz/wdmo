\begin{problem}{1}
	Wykazać, że dodatnia liczba całkowita $n$ ma co najwyżej $2\sqrt{n}$ dzielników.
\end{problem}

\begin{problem}{2}
	Liczbę nazwiemy \textit{wielodzielną}, jeśli ma co najmniej $1000$ dzielników. Rozstrzygnąć, czy istnieje taka liczba całkowita $n$, że liczby
	\[
		n, n + 1, n + 2, ..., n + 1000
	\]
	są wielodzielne.
\end{problem}

\begin{problem}{3}
	Niech $d(n)$ oznacza liczbę dodatnich dzielników liczby $n$ -- włączając liczby $1$ i $n$. Wykazać, że istnieje nieskończenie wiele dodatnich liczb całkowitych $n$, dla których liczba $\frac{n}{d(n)}$ również jest całkowita.
\end{problem}

\begin{problem}{4}
	Dana jest nieparzysta liczba pierwsza $p$. Obliczyć wartość wyrażenia
	\[
		\left(\frac{0}{p}\right) + 
		\left(\frac{1}{p}\right) + 
		... +
		\left(\frac{p - 1}{p}\right).
	\]
\end{problem}

\begin{problem}{5}
	Niech $p$ będzie liczbą pierwszą dającą resztę $2$ z dzielenia przez $3$. Dla pewnych liczb całkowitych $a$, $b$ liczba $a^3 - b^3$ jest podzielna przez $p$. Wykazać, że liczba $a - b$ również jest podzielna przez $p$.
\end{problem}


\begin{problem}{6}
	Niech $a_1, ..., a_n$ oraz $b_1, ..., b_n$ będą dodatnimi liczbami całkowitymi, że dla każdej liczby całkowitej $n - 1\geqslant i \geqslant 0$ zachodzi nierówność $b_{i + 1} \geqslant 2b_{i}$. Wykazać, że istnieje nieskończenie wiele liczb całkowitych $k$, że nie zachodzi żadne z przystawań
	\begin{gather*}
		k \equiv a_1 \pmod{b_1}, \\
		k \equiv a_2 \pmod{b_2}, \\
		.., \\
		k \equiv a_n \pmod{b_n}.
	\end{gather*}
\end{problem}

