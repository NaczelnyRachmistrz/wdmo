\newpage
\hints{Indukcja matematyczna}

\begin{hints_list}

	\item Rozpatrz trójkąt tworzony przez trzy kolejne wierzchołki $n$-kąta.

	\item Odejmij stronami tezę zadania dla $n + 1$ i $n$.

	\item
	Z założenia indukcyjnego możemy przenieść wszystkie dyski, poza najniżej położonym, na drugą igłę. Należy zauważyć, że dysk, którego nie używamy nie przeszkodzi w wykonaniu takiego przełożenia.

	\item Co mówi założenie indukcyjne? Rozpatrz przypadek, gdy z wyróżnionego punktu wychodza krawędzie różnych kolorów.

	\item Wykaż, że dla każdej liczby $n$ zachodzi równość $a_{n + 1} = 1 - a_1a_2a_3\cdot ... \cdot a_{n}.$.

	\item Podziel planszę na cztery części na pomocą dwóch prostych.

	\item Usuń dwie liczby i podziel na dwa zbiory o równej sumie elementów.
\end{hints_list}