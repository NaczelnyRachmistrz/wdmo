\newpage
\begin{center}
	\textbf{Indukcja matematyczna}
\end{center}

\begin{enumerate}
	\item Zauważmy, że dla $n = 3$ teza jest znanym faktem -- mianowicie suma kątów w trójkącie wynosi $180\degree$.


	\begin{center}
		\begin{tikzpicture}[scale=0.6]
			\tkzDefPoint(0,0){A}
			\tkzDefPoint(4,0){B}
			\tkzDefPoint(6,1){C}
			\tkzDefPoint(5,3){D}
			\tkzDefPoint(3,3){E}
			\tkzDefPoint(-1,2){F}
			\tkzDrawSegments(A,B B,C C,D D,E E,F F,A)
			\tkzDrawSegments[dashed](B,D)
			\tkzDrawPoints(A,B,C,D,E,F)
		\end{tikzpicture}
	\end{center}


	Załóżmy, że dla każdego $n$-kąta wypukłego suma jego kątów wewnętrznych wynosi $(n - 2) \cdot 180\degree$. Rozpatrzmy dowolny $n+1$-kąt wypukły. Zauważmy, że ma on więcej niż trzy wierzchołki, więc możemy ,,odciąć'' trójkąt złożony z trzech kolejnych wierzchołków. Podzielimy w ten sposób $n + 1$ kąt na $n$-kąt i trójkąt. Korzystając z wypukłości rozpatrywanego wielokąta możemy zauważyć, że suma miar jego kątów wewnętrznych jest sumą miar kątów obu tych wielokątów. Wynosi więc ona
	\[
		(n - 2) \cdot 180\degree + 180\degree = (n - 1) \cdot 180\degree,
	\]
	czego należało dowieść.

	\item Sprawdzamy, że dla $n = 1$ postulowana równość zachodzi.

	Załóżmy, że równość
	\[
		1^2 + 2^2 + 3^2 + ... + n^2 = \frac{n(n + 1)(2n + 1)}{6}
	\]
	zachodzi dla pewnej liczby $n$. Chcemy wykazać tezę dla $n + 1$, czyli
	\[
		1^2 + 2^2 + 3^2 + ... + n^2 + (n + 1)^2 = \frac{(n + 1)(n + 2)(2n + 3)}{6}.
	\]
	Zauważmy, że sprowadza się ona do wykazania toższamości
	\[
		\frac{n(n + 1)(2n + 1)}{6} + (n + 1)^2 = \frac{(n + 1)(n + 2)(2n + 3)}{6}.
	\]
	Przekształcając powyższą równość równoważnie otrzymujemy kolejno
	\begin{align*}
		n(n + 1)(2n + 1) + 6(n + 1)^2 &= (n + 1)(n + 2)(2n + 3), \\
		2n^3 + 3n^2 + n + 6(n + 1)^2 &= 2n^3 + 9n^2 + 13n + 6, \\
		6(n + 1)^2 &= 6n^2 + 12n + 6, \\
		(n + 1)^2 &= n^2 + 2n + 1.
	\end{align*}
	Prawdziwość ostatniej równości dowodzi tezy.

	\item
	Tezę wykażemy indukcją po $n$. Zauważmy, że dla $n = 1$ teza jest oczywista -- wystarczy po prostu przełożyć dysk na trzecią igłę.

	Załóżmy, że jesteśmy w stanie przełożyć $n - 1$ dysków z pierwszej igły na trzecią. Możemy oczywiście zauważyć, że jest to równoważne chociażby możliwości przełożenia ich z igły pierwszej na drugą.

	Przełożenia $n$ dysków dokonujemy w następujący sposób:

	\begin{enumerate}
		\item Przekładamy $n - 1$ dysków z góry pierwszej igły na drugą igłę. Zauważmy, że dysk o największym rozmiarze nie przeszkadza nam skorzystać z założenia indukcyjnego, gdyż nie uniemożliwi on wykonania żadnego ruchu.

		\item Dysk pozostawiony na pierwszej igle przekładamy na igłę ostatnią.

		\item Przekładamy $n - 1$ dysków z drugiej igły na trzecią. Analogicznie zauważamy, że obecność jednego dysku na trzeciej igle nie jest problemem.
	\end{enumerate}

	\item Dla $n = 3$ mamy trójkąt. Wybierając kolor, na który pomalowano co najmniej dwa odcinki odcinki, postulowana własność będzie spełniona.

	Załóżmy, że dla teza zachodzi dla $n$ punktów. Rozpatrzmy zbiór $n + 1$ punktów. Wyróżnijmy pewien punkt $P$. Punktów poza $P$ jest dokładnie $n$, więc na mocy założenia istnieje kolor -- bez straty ogólności czerwony -- że pomiędzy kążdymi dwoma punktami poza $P$ istnieje łamana tego koloru. 

\begin{minipage}{0.5\textwidth}
\begin{center}
	\begin{tikzpicture}
    \tkzDefPoint(3,3){P}
    \tkzDefPoint(0,2){v_1}
    \tkzDefPoint(1.5,0){v_2}
    \tkzDefPoint(3,0){v_3}
    \tkzDefPoint(5,1){v_4}
    \tkzDrawPoints(P, v_1,v_2,v_3,v_4)
    \tkzDrawSegments[dashed](v_1,v_2 v_2,v_3 v_3,v_4 v_1,v_4)
    \tkzDrawSegments(v_1,v_3 v_2,v_4)
    \tkzDrawSegments(P,v_2 P,v_3 P,v_4 v_1,P)
    \tkzLabelPoints[above](P)
	\end{tikzpicture}\\
	
\end{center}
\end{minipage}
\begin{minipage}{0.5\textwidth}
\begin{center}
	\begin{tikzpicture}
    \tkzDefPoint(3,3){P}
    \tkzDefPoint(0,2){v_1}
    \tkzDefPoint(1.5,0){v_2}
    \tkzDefPoint(3,0){v_3}
    \tkzDefPoint(5,1){v_4}
    \tkzDrawPoints(P, v_1,v_2,v_3,v_4)
    \tkzDrawSegments[dashed](v_1,v_2 v_2,v_3 v_3,v_4 v_1,v_4 P,v_4)
    \tkzDrawSegments(v_1,v_3 v_2,v_4)
    \tkzDrawSegments(P,v_2 P,v_3  v_1,P)
    \tkzLabelPoints[above](P)
	\end{tikzpicture}\\
\end{center}
\end{minipage}
\begin{center}
	Na rysunku zamiast kolorów użyto podziału na linię ciągła i przerywaną.
\end{center}

	Rozpatrzmy dwa przypadki:
	\begin{enumerate}
		\item Punkt $P$ jest połączony czerwoną krawędzią z pewnym innym punktem $Q$. Wówczas wybierając dowolny punkt $X$, na mocy założenia wiemy, że istnieje czerwona ścieżka między $X$ i $Q$. Dokładając do niej odcinek między $P$ i $Q$ otrzymujemy ścieżkę między $P$ oraz $X$.
		Wykazaliśmy, że istnieje ścieżka między punktem $P$ i każdym innym punktem. Łącząc to z faktem, że na mocy założenia indukcyjnego taka ścieżka istnieje między każdą inną parą punktów, otrzymujemy, że dla koloru czerwonego teza jest spełniona.
		\item Punkt $P$ jest połączony z każdym innym punktem niebieskim odcinkiem. Wówczas łatwo zauważyć, że pomiędzy każda parą punktów możemy przejść jednym albo dwoma niebieskimi odcinkami przechodzącymi przez punkt $P$.
	\end{enumerate}

	\item Na początku wykażemy indukcyjnie, że dla każdego $n$ zachodzi równość
	\[
		a_{n + 1} = 1 - a_1a_2a_3\cdot ... \cdot a_{n}.
	\]

	Równośc dla $n = 0$ zachodzi na mocy założeń.

	Załóżmy, że
	\[
		a_{n} = 1 - a_1a_2a_3\cdot ... \cdot a_{n - 1}.
	\]
	Skoro $a_{n + 1} = 1 - a_n(1 - a_n)$, to otrzymujemy
	\[
		a_{n + 1} = 1 - a_n(1 - a_n) = 1 - a_n \cdot a_1a_2a_3\cdot ... \cdot a_{n - 1} = 1 - a_1a_2a_3\cdot ... \cdot a_{n}.
	\]

	Wieć na mocy zasady indukcji matematycznej postulowana równość zachodzi.

	Teraz przejdziemy do udowodnienia tezy.

	Dla $n = 1$ jest ona oczywista.

	Załóżmy, że zachodzi równość
	\[
		a_1a_2a_3...a_n\left(\frac{1}{a_1} + \frac{1}{a_2} + \frac{1}{a_3} + ... + \frac{1}{a_n}\right) = 1.
	\]
	Chcemy wykazać, że
	\[
		a_1a_2a_3...a_na_{n+1}\left(\frac{1}{a_1} + \frac{1}{a_2} + \frac{1}{a_3} + ... + \frac{1}{a_n} + \frac{1}{a_{n + 1}}\right) = 1.
	\]
	Przekształcamy powyższą równość korzystając z założenia
	\begin{multline*}
		a_1a_2a_3...a_na_{n+1}\left(\frac{1}{a_1} + \frac{1}{a_2} + \frac{1}{a_3} + ... + \frac{1}{a_n} + \frac{1}{a_{n + 1}}\right) = a_{n+1} \cdot a_1a_2a_3...a_n\left(\frac{1}{a_1} + \frac{1}{a_2} + \frac{1}{a_3} + ... + \frac{1}{a_n}\right) + \\ + a_1a_2a_3...a_n = a_{n + 1} + a_1a_2a_3...a_n = 1 - a_1a_2a_3...a_n + a_1a_2a_3...a_n = 1.
	\end{multline*}


	\item Zauważmy, że plansza $2\times2$ z usuniętym rogiem jest w istocie L-klockiem, więc da się ją pokryć.

	\begin{center}
		\begin{tikzpicture}[scale=0.3]
	    \tkzDefPoint(0,0){A}
	    \tkzDefPoint(0,10){B}
	    \tkzDefPoint(10,10){C}
	    \tkzDefPoint(10,0){D}
	    \tkzDefPoint(9,0){P1}
	    \tkzDefPoint(9,1){P2}
	    \tkzDefPoint(10,1){P3}
	    \tkzDrawSegments(A,B B,C C,D D,A)
	    \tkzDrawSegments(P2,P1 P3,P2)


	    \tkzDefPoint(5,0){G1}
	    \tkzDefPoint(5,10){G2}
	    \tkzDefPoint(0,5){G3}
	    \tkzDefPoint(10,5){G4}
	    \tkzDrawSegments(G1,G2 G3,G4)

	    \tkzDefPoint(4,4){T1}
	    \tkzDefPoint(4,6){T2}
	    \tkzDefPoint(6,6){T3}
	    \tkzDefPoint(6,5){T4}
	    \tkzDefPoint(5,4){T5}
	    \tkzDrawSegments(T1,T2 T2,T3 T3,T4 T1,T5) 
		\end{tikzpicture}\\
	\end{center}

	Załóżmy, że dla planszy $2^{n - 1} \times 2^{n - 1}$ istnieje szukane pokrycie. Pokrycie dla planszy $2^{n} \times 2^{n}$ kontruujemy następująco. Dzielimi plansze dwiema prostymi na trzy jednakowe części i czwartą taką samą, tylko bez rok. Kładziemy jeden klocek na środku tak jak na rysunku. Wówczas plansza jest podzielona na cztery jednakowe puste części, które na mocy założenia indukcyjnego można pokryć.

	\item 
	Przez multizbiór rozumiemy zbiór w którym jeden element może występować kilka razy.

	Załóżmy, że $x_1 \geqslant x_2 \geqslant ... \geqslant x_n$. 

	Wykażemy tezę dla $n = 3$. Podział na zbiory $\{x_1 - x_2, x_2 - x_3\}$ oraz $\{x_1 - x_3\}$ spełnia warunki zadania.

	Załóżmy, że teza zachodzi dla $2n + 1$, wykażemy ją dla $2n + 3$.
	Rozpatrzmy szukany podział multizbioru różnic zbioru $\{x_1, x_2, ..., x_{2n + 1}\}$ na multizbiory $A$ i $B$ o równej sumie elementów.

	Dorzucamy do multizbioru $A$ liczby
	\[
		({x_{2k+3}-x_{2k+1},x_{2k+3}-x_{2k},\cdots,x_{2k+3}-x_{k+2}}) \thickspace {\cup}\thickspace ({x_{2k+2}-x_{k+1},\cdots,x_{2k+2}-x_1}),
	\]
	a do multizbioru $B$ liczby
	\[
		(x_{2k+3}-x_{2k+2})\thickspace {\cup}\thickspace ({x_{2k+2}-x_{2k+1},x_{2k+2}-x_{2k},\cdots,x_{2k+3}-x_{k+2}})\thickspace {\cup} \thickspace({x_{2k+3}-x_{k+1},\cdots,x_{2k+3}-x_1}).
	\]
	Łatwo sprawdzić, że suma dorzuconych elementów jest równa.
\end{enumerate}