\newpage
\hints{Indukcja matematyczna}

\begin{hints_list}
	\item Przeprowadź rozumowanie indukcyjne po liczbie wierzchołków $n$.

	\item Sprawdź, że równość zachodzi dla $n = 1$. Załóż, że równość zachodzi dla $n$ i spróbuj wykazać ją dla $n + 1$.

	\item Przeprowadź indukcję po liczbie $n$. Skorzystaj dla wszystkich początkowych dysków poza najniżej położonym.

	\item Rozpatrz $n + 1$ punktów i zobacz, co się stanie, jeśli usuniemy jeden z nich.

	\item Spróbuj wykazać tezę inducją po $n$. Aby to zrobić, trzeba będzie wykazać indukcyjnie inną równość pomocniczą.

	\item Spróbuj podzielić planszę $2^{n + 1} \cdot 2^{n + 1}$ na kilka części.

	\item Wykaż, że jeśli teza zachodzi dla $n$, to zachodzi również dla $n + 2$.
\end{hints_list}