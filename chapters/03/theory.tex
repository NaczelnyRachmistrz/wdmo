% Rozdział 3 – Bijekcje i bajki kombinatoryczne

\theory{Bijekcje i bajki kombinatoryczne}

\noindent
W tym rozdziale będziemy analizować różne zbiory i relacje między nimi. W części zadań trzeba będzie pokazać, że pewne dwa zbiory mają tyle samo elementów. Jedną z metod dowodzenia tego typu stwierdzeń jest połączenie elementów danych zbiorów w pary. Takie przyporządkowanie nazywamy \textit{bijekcją}.

\begin{center}
    \begin{tikzpicture}
        \tkzDefPoint(-1,3){P_0}
        \tkzDefPoint(0,3){P_1}
        \tkzDefPoint(1,3){P_2}
        \tkzDefPoint(2,3){P_3}
        \tkzDefPoint(3,3){P_4}
        \tkzDefPoint(4,3){P_5}
        \tkzDefPoint(5,3){P_6}


        \tkzDefPoint(-1,1){Q_0}
        \tkzDefPoint(0,1){Q_1}
        \tkzDefPoint(1,1){Q_2}
        \tkzDefPoint(2,1){Q_3}
        \tkzDefPoint(3,1){Q_4}
        \tkzDefPoint(4,1){Q_5}
        \tkzDefPoint(5,1){Q_6}

        \tkzDrawSegments[dashed](P_1,Q_1 P_2,Q_2 P_3,Q_3 P_4,Q_4 P_5,Q_5 P_6,Q_6)

        \tkzDrawPoints(P_1, P_2, P_3, P_4, Q_1, Q_2, Q_3, Q_4, P_5, P_6, Q_5, Q_6)

        \tkzLabelPoint[left](P_0){Zbiór $A$}
        \tkzLabelPoint[left](Q_0){Zbiór $B$}
    \end{tikzpicture}
\end{center}

\vspace{10px}

\noindent
Aby stwierdzić czy przyporządkowanie jest bijekcją, wystarczy sprawdzić czy każdy element jednego zbioru jest przyporządkowany do \textit{dokładnie} jednego elementu drugiego zbioru. Poniżej dwa przykłady przyporządkowań, które nie są bijekcjami.

\begin{center}
    \begin{tikzpicture}[scale=0.5]
        \tkzDefPoint(-1,3){P_0}
        \tkzDefPoint(0,3){P_1}
        \tkzDefPoint(1,3){P_2}
        \tkzDefPoint(2,3){P_3}
        \tkzDefPoint(3,3){P_4}
        \tkzDefPoint(4,3){P_5}
        \tkzDefPoint(5,3){P_6}


        \tkzDefPoint(-1,1){Q_0}
        \tkzDefPoint(0,1){Q_1}
        \tkzDefPoint(1,1){Q_2}
        \tkzDefPoint(2,1){Q_3}
        \tkzDefPoint(3,1){Q_4}
        \tkzDefPoint(4,1){Q_5}
        \tkzDefPoint(5,1){Q_6}

        \tkzDrawSegments[dashed](P_1,Q_1 P_2,Q_2 P_3,Q_3 P_4,Q_4 P_5,Q_5 P_6,Q_5 P_1,Q_2)

        \tkzDrawPoints(P_1, P_2, P_3, P_4, Q_1, Q_2, Q_3, Q_4, P_5, P_6, Q_5)

        \tkzLabelPoint[left](P_0){Zbiór $A$}
        \tkzLabelPoint[left](Q_0){Zbiór $B$}
    \end{tikzpicture}
    \hspace{20px}
    \begin{tikzpicture}[scale=0.5]
        \tkzDefPoint(-1,3){P_0}
        \tkzDefPoint(0,3){P_1}
        \tkzDefPoint(1,3){P_2}
        \tkzDefPoint(2,3){P_3}
        \tkzDefPoint(3,3){P_4}
        \tkzDefPoint(4,3){P_5}
        \tkzDefPoint(5,3){P_6}


        \tkzDefPoint(-1,1){Q_0}
        \tkzDefPoint(0,1){Q_1}
        \tkzDefPoint(1,1){Q_2}
        \tkzDefPoint(2,1){Q_3}
        \tkzDefPoint(3,1){Q_4}
        \tkzDefPoint(4,1){Q_5}
        \tkzDefPoint(5,1){Q_6}

        \tkzDrawSegments[dashed](P_1,Q_1 P_2,Q_2 P_3,Q_3 P_4,Q_4 P_5,Q_5)

        \tkzDrawPoints(P_1, P_2, P_3, P_4, Q_1, Q_2, Q_3, Q_4, P_5, Q_5, Q_6)

        \tkzLabelPoint[left](P_0){Zbiór $A$}
        \tkzLabelPoint[left](Q_0){Zbiór $B$}
    \end{tikzpicture}
\end{center}

\heading{Przykład 1}

\noindent
Dla pewnej liczby całkowitej $n$ jej \textit{podziałem} nazwiemy takie dodatnie liczby całkowite $(a_1, ..., a_t)$, że
\begin{gather*}
    n = a_1 + a_2 + ... + a_t \\
    a_1 \geqslant a_2 \geqslant a_3 \geqslant ... \geqslant a_t \geqslant 1.
\end{gather*}

\noindent
Niech $n$, $k$ będą dodatnimi liczbami całkowitymi. Wykazać, że liczba podziałów $n$, które składają się dokładnie z $k$ liczb jest równa liczbie podziałów $n$, takich, że największy składnik każdego z nich jest równy dokładnie $k$.

\vspace{5px}

\heading{Rozwiązanie}

\vspace{5px}

\noindent
Weźmy dowolny podział liczby $n$. Niech $n = a_1 + a_2 + ... + a_t$. Rozpatrzmy jego reprezentację graficzną zwaną diagramem Ferrera.
Poniżej narysowano diagram Ferrera dla podziału $12 = 4 + 4 + 3 + 1$.
W każdym kolejnym wierszu znajduje się tyle kropek, ile wynosi kolejny składnik z podziału. 

\begin{center}
    \begin{tikzpicture}[scale=0.6]

        \tkzDefPoint(-1,3){A_1}
        \tkzDefPoint(0,3){P_1}
        \tkzDefPoint(1,3){P_2}
        \tkzDefPoint(2,3){P_3}
        \tkzDefPoint(3,3){P_4}


        \tkzDefPoint(-1,2){A_2}
        \tkzDefPoint(0,2){P_5}
        \tkzDefPoint(1,2){P_6}
        \tkzDefPoint(2,2){P_7}
        \tkzDefPoint(3,2){P_8}

        \tkzDefPoint(-1,1){A_3}
        \tkzDefPoint(0,1){P_9}
        \tkzDefPoint(1,1){P_10}
        \tkzDefPoint(2,1){P_11}

        \tkzDefPoint(-1,0){A_4}
        \tkzDefPoint(0,0){P_12}

        \tkzDrawPoints(P_1, P_2, P_3, P_4, P_5, P_6, P_7, P_8, P_9, P_10, P_11, P_12)
        \tkzLabelPoint[left](A_1){4}
        \tkzLabelPoint[left](A_2){4}
        \tkzLabelPoint[left](A_3){3}
        \tkzLabelPoint[left](A_4){1}

        \tkzDefPoint(0,4){B_1}
        \tkzDefPoint(1,4){B_2}
        \tkzDefPoint(2,4){B_3}
        \tkzDefPoint(3,4){B_4}
        \tkzLabelPoint[above](B_1){4}
        \tkzLabelPoint[above](B_2){3}
        \tkzLabelPoint[above](B_3){3}
        \tkzLabelPoint[above](B_4){2}
    \end{tikzpicture}
\end{center}

\noindent
Zastanówmy się, co znaczą założenia zadania w języku rozpatrywanych diagramów.
Jeśli w podziale jest dokładnie $k$ liczb, to diagram Ferrera będzie składał się dokładnie z $k$ wierszy. Jeśli  największy składnik podziału jest równy $k$, to kolumn będzie dokładnie $k$.

\vspace{10px}
\noindent
Zauważmy, że patrząc na dowolny diagram Ferrera ,,od góry'' -- traktujemy kolumny jako wiersze i vice versa -- otrzymamy inny diagram Ferrera. W podanym przykładzie z podziału $12 = 4 + 4 + 3 + 1$ otrzymamy w ten sposób podział $12 = 4 + 3 + 3 + 2$.

\vspace{10px}
\noindent
Jeśli diagram Ferrera przedstawiał podział $n$, który składa się dokładnie z $k$ liczb, to podział otrzymany w powyższy sposób ma największy składnik każdego z nich równy dokładnie $k$. Obie z tych własności są równoważne temu, że diagram ma $k$ wierszy.

\vspace{10px}
\noindent
Powyższe przyporządkowanie łączy elementy danych w zadaniu zbiorów w pary -- dokładnie jeden podział pierwszego rodzaju z dokładnie jednym podziałem drugiego rodzaju. Rysując diagram dla pewnego podziału, otrzymamy dokładnie jeden podział z drugiego zbioru, więc to parowanie jest dobre. Stąd wynika, że rozpatrywane zbiory mają tyle samo elementów.

\qed

\vspace{10px}

\noindent
Pokazaliśmy, że pewne dwa zbiory mają tę samą liczbę elementów. Teraz spróbujemy za pomocą kombinatoryki udowodnić równość algebraiczną.

\vspace{5px}


\heading{Przykład 2}

\noindent
Wykazać, że dla wszystkich dodatnich liczb całkowitych $n$, $k$ zachodzi równość
\[
    \sum^{n}_{k=0} {{n}\choose{k}} 2^k = 3^n.
\]

\vspace{5px}

\heading{Rozwiązanie}

\vspace{5px}

\noindent
Na imprezę przyszło $n$ matematyczek. Każda z nich wzięła kapelusz, czapkę lub przyszła bez okrycia głowy. Obliczmy ile różnych wariantów nakryć głowy mogło się zdarzyć na dwa sposoby.
\begin{enumerate}
    \item Każda z dziewczyn mogła wybrać jedną z trzech opcji ubioru, było ich $n$, więc liczba możliwości wynosi $3^n$.
    \item Przyjmijmy, że $n - k$ dziewczyn nie przyniosło żadnego nakrycia głowy. 
    Wówczas możemy wybrać te dziewczyny na ${{n}\choose{n - k}} = {{n}\choose{k}}$ sposobów. 
    Następnie każda z pozostałych $k$ dziewczyn wybrała jedno z dwóch dostępnych nakryć głowy. 
    Więc mogą to zrobić na $2^k$ sposobów. Z reguły mnożenia wynika, że dla ustalonej liczby $k$ jest dokładnie ${{n}\choose{k}} 2^k$ wariantów. Sumując po wszystkich $k$ otrzymujemy $\sum^{n}_{k=0} {{n}\choose{k}} 2^k$.
\end{enumerate}

\noindent
Obliczając jedną rzecz na dwa sposoby otrzymaliśmy liczby, które muszą być równe.

\qed

\vspace{10px}

\noindent
Rozumowania podobne do powyższego nazywane są bajkami kombinatorycznymi.

\vspace{10px}