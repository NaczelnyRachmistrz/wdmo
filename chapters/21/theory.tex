% Rozdział 21 - Maksima, niezmienniki i procesy

\theory{Maksima, niezmienniki i procesy}

\heading{Przykład 1}

%Source: http://web.mit.edu/yufeiz/www/wc08/invariants.pdf Sample problem 2
\noindent
W ciągu $(1,\; 0,\; 1,\; 0,\; 1,\; 0,\; 3,\; 5,\; ...)$ każdy element, począwszy od siódmego jest równy sumie sześciu poprzedzających go elementów. Wykazać, że ten ciąg nie zawiera spójnego podciągu $(0, \;1, \;0, \;1,\; 0,\;1)$.

\heading{Rozwiązanie}

%Source: https://alexanderrem.weebly.com/uploads/7/2/5/6/72566533/miscellaneousproblems.pdf
\heading{Przykład 2}

\noindent
Na tablicy zapisano dwie dodatnie i różne liczby całkowite. W każdym ruchu, przed którym na tablicy znajdowały się liczby $a > b$, mniejsza z obu liczb jest zmazywana i zastępowana liczbą 
\[	
	\frac{ab}{|a - b|}.
\]
Wykazać, że ten proces zakończy się po skończonej liczbie ruchów.\\

\heading{Rozwiązanie}

% to da się skończyć nie mówiąc, że to jest algorytm Euklidesa

\heading{Algorytm Euklidesa}