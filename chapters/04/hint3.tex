\hints{Liczby pierwsze i reszty z dzielenia}


\begin{hints_list}
	\item Podstaw $n = k(p - 1)$. Zauważ, że $2^n \equiv 1 \pmod{p}$. 
	\item Zauważ, że zbiory $\{1, \; 2, \; 3, \; ..., \; p - 1\} $ i $\{1^{-1},\; 2^{-1},\; 3^{-1},\; ..., \;(p - 1)^{-1}\}$ są sobie równe. Stąd suma ich elementów jest równa.
	\item Zauważ, że $\frac{1}{2} + \frac{1}{3} + \frac{1}{6} = 1$.
	\item Skoro wspomniane parowanie istnieje, to 
	$
		2 \cdot 3 \cdot ... \cdot (p - 2) \equiv 1 \pmod{p}.
	$,
	bo możemy podzielić te liczby na pary, z których każda zredukuje się do liczby $1$.
\end{hints_list}