\begin{problem}{1} 
	Dana jest liczba pierwsza $p$. Udowodnić, że istnieje taka liczba całkowita $n$, że 
	\[
		2^n \equiv n \pmod{p}.
	\]
\end{problem}

\begin{problem}{2}
	Dana jest liczba pierwsza $p\geqslant 3$. Niech 
	\[
		1 + \frac{1}{2} + \frac{1}{3} + ... + \frac{1}{p-1} = \frac{a}{b}
	\]
	dla pewnych dodatnich liczb całkowitych $a$, $b$.
	Udowodnić, że $p\big| a$.
\end{problem}

\begin{problem}{3}
	Udowodnij, że istnieje $n$, dla którego $2^n+3^n+6^n\equiv 1 \pmod{p}.$
\end{problem}

\begin{problem}{4}
	Wykazać, że zachodzi przystawanie
	\[
		(p - 1)! \equiv -1 \pmod{p}.
	\]
\end{problem}