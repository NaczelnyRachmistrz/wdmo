%Source: https://om.mimuw.edu.pl/static/app_main/problems/om71_1r.pdf P3
\begin{problem}{1}
	Szachownicę o wymiarach $15 \times 15$ przykryto przy pomocy płytek o wymiarach $2 \times 2$ i $3 \times 3$ w taki sposób, że każde pole jest przykryte przez dokładnie jedną płytką oraz płytki nie wystają poza szachownicę. Wyznaczyć najmniejszą liczbę użytych płytek $3 \times 3$, dla której jest to możliwe.
\end{problem}

%Source: Marcin Pitera ,,Kolorowe kwadraty'' 4.2
\begin{problem}{2}
	Płaszczyznę pokolorowano trzema kolorami. Wykazać, że istnieje odcinek o długości $1$, który ma końce tego samego koloru.
\end{problem}

%Source: Marcin Pitera ,,Kolorowe kwadraty'' 5.6
\begin{problem}{3}
	Sad podzielony jest na $100$ przystających kwadratów, tworzących szachownicę $10 \times 10$. Na dziewieciu kwadratach znajduje się roślina. Co roku roślina rozrasta się na pola, które już sąsiadują -- mają wspólny bok -- z co najmniej dwoma polami, na których już rośnie roślina. Udowodnić, że roślina nigdy nie rozrośnie się tak, aby zajmować cały sad.
\end{problem}
 
%Source: MBL QQ
\begin{problem}{4}
	Każdy punkt płaszczyzny pokolorowano jednym z dwóch kolorów. Wykazać, że istnieje taki trójkąt o kątach $80\degree$, $80\degree$ i $20\degree$, że wszystkie jego wierzchołki są jednego koloru.
\end{problem}

%Source: Marcin Pitera ,,Kolorowe kwadraty'' Z 1.1
\begin{problem}{5}
	Wewnątrz szachownicy o wymiarach $19 \times 19$ rozważmy mniejszą szachownicę o wymiarach $16 \times 16$. Rozstrzygnąć, czy można tak umieścić klocki $1 \times 5$ wewnątrz większej szachownicy, aby w całości pokryły mniejszą szachownicę.
\end{problem}

%Source: Marcin Pitera ,,Kolorowe kwadraty'' Z 1.5
\begin{problem}{6}
	Czy można wypełnić szachownicę o wymiarach $8 \times 8$ przy pomocy jednego $Z$-klocka oraz dowolnej liczby prostokątów $1 \times 4$.
\end{problem}


