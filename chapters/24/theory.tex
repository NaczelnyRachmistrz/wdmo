% Rozdział 24 – Szachownice, klocki i kolorowanie

\theory{Szachownice, klocki i kolorowanie}

% jakieś zadania z planszą

%Source: https://om.mimuw.edu.pl/static/app_main/problems/om70_1r.pdf
\heading{Przykład 1}

\noindent
Szachownicę o wymiarach $2018\times 2018$ przykryto przy pomocy jednej kwadratowej płytki o wymiarach $2 \times 2$ oraz $\frac{2018^2 − 4}{5}$ prostokątnych płytek o wymiarach $1 \times 5$ w taki sposób, że każde pole szachownicy jest przykryte przez dokładnie jedną płytkę (płytki można obracać). Wykazać, że płytka 2 × 2 nie przykrywa żadnego pola o krawędzi zawartej w brzegu szachownicy.

\heading{Rozwiązanie}

% Source: https://artofproblemsolving.com/community/c6h597130p3543398
\heading{Przykład 2}

\noindent
Dany jest graf. W każdym ruchu na aktualnym grafie $G$ można wykonać jedną z dwóch operacji

(i) Jeśli pewien wierzchołek ma nieparzysty stopień, to usunąć wszystkie krawędzie, które z niego wychodzą.

(ii) Można dodać do grafu $G$ graf $G'$ będący kopią grafu $G$. Następnie łączymy odpowiadające sobie wierzchołki z $G$ oraz $G'$. 

\noindent
Wykazać, że niezależnie od grafu początkowego, można wykonać pewną liczbę operacji, aby otrzymać graf bez żadnej krawędzi.

%rysunek

\heading{Rozwiązanie}


% napisać, żeby nie zamykali się w ideach, poszukiwali nowych pomysłów
%napisać o lustrzanyc odbiciach klocków
