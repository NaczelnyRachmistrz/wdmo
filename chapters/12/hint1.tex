\hints{Bardziej zaawansowane nierówności}

\begin{hints_list}
	\item Skorzystaj z nierówności Schwarza w formie Engela.
	\item Rozpatrzmy ciągi $(a^2, b^2, c^2)$ oraz $bc$, $ac$, $ab$.
	\item Przyda się jedynka trygonometryczna, jak i fakt, że sinus i cosinus przyjmują wartości o module mniejszym od 1.
	\item Skorzystaj z nierówności o ciągach jednomonotonicznych.
	\item Zobacz co się stanie, gdy podstawisz $a = 0$.
\end{hints_list}