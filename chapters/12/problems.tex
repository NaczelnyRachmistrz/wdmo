%Source: Wędrówki po krainie nierówności 3.5 przykład 2
\begin{problem}{1}
	Dane są liczby dodatnie $a$, $b$, $c$. Wykaż, że zachodzi nierównośc
	\[
		\frac{a}{2b + c} + \frac{b}{2c + a} + \frac{c}{2a + b} \geqslant 1. 
	\]
\end{problem}

%Source: Wędrówki po krainie nierówności 3.5 przykład 3
\begin{problem}{2}
	Udowodnić, że dla dowolnych liczb dodatnich $a$, $b$, $c$ zachodzi nierówność
	\[
		a^3b + b^3c + c^3a \geqslant a^2bc + ab^2c + abc^2.
	\]
\end{problem}

%Source: Wędrówki po krainie nierówności 3.5.3
\begin{problem}{3}
	Udowodnić, że dla dowolnych liczb rzeczywistych $\alpha_1, \; \alpha_2, \; \alpha_n$, gdzie $n \geqslant 2$ zachodzi nierówność
	\[
		\sin{\alpha_1}\cdot ... \cdot \sin{\alpha_n} + \cos{\alpha_1}\cdot ... \cdot \cos{\alpha_n} \leqslant 1.
	\]
\end{problem}

%Source: well-known
\begin{problem}{4}
	Dane są liczby rzeczywiste $a_1 \geqslant a_2 \geqslant ... \geqslant a_n$ oraz $b_1 \geqslant b_2 \geqslant ... \geqslant b_n$. Wykazać, że zachodzi nierówność
	\[
		n(a_1b_1 + a_2b_2 + ... + a_nb_n) \geqslant (a_1 + a_2 + ... + a_n)(b_1 + b_2 + ... + b_n).
	\]
\end{problem}

%Source: https://omj.edu.pl/uploads/attachments/omj2017-tresci.pdf
% Obóz OMJ 2017 poziom OMJ P 21
\begin{problem}{5}
	Rozstrzygnąć, czy dla dowolnych liczb dodatnich $a$, $b$, $c$ zachodzi nierównośc
	\[
		a^2b^3 + b^2c^3 + c^2a^3 \leqslant ab^4 + bc^4 + ca^4.
	\]
\end{problem}
