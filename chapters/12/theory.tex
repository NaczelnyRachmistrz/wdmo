% Rozdział 12 – bardziej zaawansowanie nierówności

\theory{Bardziej zaawansowane nierówności}

\noindent
Na początku udowodnimy dwie bardzo znane nierówności. 

\vspace{10px}

% nierówność Cauchego–Schwarza
\heading{Nierówność Cauchego–Schwarza}

\noindent
Dla liczb rzeczywistych $a_1, a_2, ..., a_n$ oraz $b_1, b_2, ..., b_n$ zachodzi nierówność
\[
	\left(a_1^2 + a_2^2 + ... + a_n^2\right)\left(b_1^2 + b_2^2 + ... + b_n^2\right) \geqslant \left(a_1b_1 + a_2b_2 + ... + a_nb_n\right).
\]

\heading{Dowód}

\noindent
Przyjmijmy, że funkcja $f$ jest dana wzorem
\[
	f(x) = \sum^{n}_{i=1} (a_ix - b_i)^2 = \left(\sum^{n}_{i=1} a_i^2\right) x^2 - 2\left(\sum^{n}_{i=1} a_ib_i\right) x + \sum^{n}_{i=1} b_i^2.
\]
Zauważmy, że jest to suma funkcji przyjmujących wartości nieujemne, a więc sama przyjmuje tylko wartości nieujemne. Stąd też jej wyróżnik $\Delta$ będzie niedodatni
\[
	\Delta = 4\left(\sum^{n}_{i=1} a_ib_i\right)^2 - 4  \left(\sum^{n}_{i=1} a_i^2\right)  \left(\sum^{n}_{i=1} b_i^2\right) \leqslant 0,
\]
co jest równoważnie tezie.

\qed

\vspace{10px}

% ta sama w formie Engela

\heading{Nierówność Cauchego–Schwarza w formie Engela}

\noindent
Dla liczb rzeczywistych $a_1, a_2, ..., a_n$ oraz $b_1, b_2, ..., b_n$ zachodzi nierówność
\[
	\frac{a_1}{b_1} + \frac{a_2}{b_2} + ... + \frac{a_n}{b_n} \geqslant \frac{\left(a_1 + a_2 + ... + a_n\right)^2}{a_1b_1 + a_2b_2 + ... + a_nb_n}.
\]

\heading{Dowód}

\noindent
Zauważmy, że na mocy nierówności Cauchego-Schwarza prawdą jest, że
\begin{gather*}
	\left(\sum^{n}_{i = 1} \left(\sqrt{\frac{a_i}{b_i}}\right)^2 \right) \left(\sum^{n}_{i = 1} \left(\sqrt{a_ib_i}\right)^2 \right) \geqslant \left(\sum^{n}_{i = 1} \sqrt{a_ib_i} \cdot \sqrt{\frac{a_i}{b_i}} \right).
\end{gather*}
Jest to równoważnie nierówności
\[
	\left(\sum^{n}_{i = 1} \frac{a_i}{b_i} \right) \left(\sum^{n}_{i = 1} a_ib_i \right) \geqslant \left(\sum^{n}_{i = 1} a_i^2 \right),
\]
z której w oczywisty sposób wynika teza.

\qed
\newpage

\noindent

% ciągi jednomonotoniczne

\noindent
Przejdziemy teraz do nierówności między ciągami jednomonotonicznymi. Ma ona bardzo ciekawy dowód, do którego będzie potrzebny nam następujący lemat.

\vspace{10px}

\heading{Lemat 1}

\noindent
Dane są liczby rzeczywiste $a_1 \geqslant a_2$ oraz $b_1 \geqslant b_2$. Wówczas zachodzi nierówność
\[
	a_1b_1 + a_2b_2 \geqslant a_1b_2 + a_2b_1.
\] 
Jeśli zaś $a_1 \geqslant a_2$ oraz $b_1 \leqslant b_2$, to zachodzi
\[
	a_1b_1 + a_2b_2 \leqslant a_1b_2 + a_2b_1.
\] 

\heading{Dowód}

\noindent
Zauważmy, że
\[
	a_1b_1 + a_2b_2 \geqslant a_1b_2 + a_2b_1 \iff (a_1 - a_2)(b_1 - b_2) \geqslant 0,
\]
\[
	a_1b_1 + a_2b_2 \leqslant a_1b_2 + a_2b_1 \iff (a_1 - a_2)(b_1 - b_2) \leqslant 0,
\]
Teza lematu wynika wprost z założeń.

\qed

\vspace{10px}

% tutaj opis przekształceń, sortowania bąbelkowego
% ,,z tego algorytmu wynika następujące twierdzenie

\heading{Twierdzenie o ciągach jednomonotonicznych}

\noindent
Liczby $a_1 \geqslant a_2 \geqslant ... \geqslant a_n$ oraz $b_1 \geqslant b_2 \geqslant ... \geqslant b_n$ są rzeczywiste. Niech $b_1', b_2', ..., b_n'$ będzie permutacją liczb $b_1, b_2, ..., b_n$. Wówczas zachodzą nierówności
\[
	a_1b_1 + a_2b_2 + ... + a_nb_n \geqslant a_1b_1' + a_2b_2' + ... + a_nb_n' \geqslant a_1b_n + a_2b_{n - 1} + ... + a_nb_1.
\]

\heading{Dowód}

\noindent
Rozpatrzmy pewną permutację $b_1', b_2', ..., b_n'$ liczb $b_1, b_2, ..., b_n$.  Zamieńmy liczby $b'_i$ oraz~$b'_{i + 1}$ miejscami i zobaczmy co stanie się z wartością wyrażenia, nazwijmy go $W$:
\[
	a_1b_1' + a_2b_2' + ... + a_nb_n'.
\]
Składnik $a_ib_i'$ zostanie zastąpiony przez $a_ib_{i + 1}'$, zaś $a_{i + 1}b_{i + 1}'$ zostanie zastąpiony przez $a_{i + 1}b_i'$. Inne składniki pozostaną bez zmian. Skoro $a_i \geqslant a_{i + 1}$, to z Lematu~1 wynika, że
\[
	a_ib_{i + 1}' + a_{i + 1}b_i' \leqslant a_ib_i' + a_{i + 1}b_{i + 1}', \quad \text{gdy} \; b_i \geqslant b_{i + 1},
\]
\[
	a_ib_{i + 1}' + a_{i + 1}b_i' \geqslant a_ib_i' + a_{i + 1}b_{i + 1}', \quad \text{gdy} \; b_i \leqslant b_{i + 1}.
\]

\vspace{10px}

\noindent
Jeśli więc liczby $b'_i$ oraz $b'_{i + 1}$ są posortowane odwrotnie do ciągu $\{a_i\}$, to zmieniając ich kolejności możemy zwiększyć wartość wyrażenia $W$.
Wykonując wiele takich zamian, za każdym razem wartość $W$ wzrośnie. Również liczba par postaci $(b_i', b_j')$, gdzie $i > j$ oraz $b_i' < b_j'$ maleje o $1$ przy każdym przestawieniu. Stąd w pewnym momencie otrzymamy permutację, w którym nie możemy wykonać rozpatrywanej operacji. Znaczy to tyle, że $b'_i \geqslant b'_{i + 1}$ dla wszystkich możliwych wartości liczby $i$, czyli po prostu $b'_i = b_i$. Finalnie, wartość rozpatrywanego wyrażenia $W$ jest równa
\[
	a_1b_1 + a_2b_2 + ... + a_nb_n.
\]
Skoro przy każdym ruchu wartość tego wyrażenia rosła, to jego finalna wartość jest większa od wartości początkowej.

\vspace{10px}

\noindent
Dowodzi to jednej z dwóch postulowanych nierówności. Dowód drugiej z nich jest analogiczny, tylko zamieniamy miejscami $b_i'$ posortowane jednakowo jak ciąg $\{a_i\}$.

\qed

\vspace{10px}

\noindent
Przyjmijmy, że ciągi $(a_1, a_2, ..., a_n)$ oraz $(b_1, b_2, ..., b_n)$ są \textit{jednakowo monotoniczne}, gdy dla dowolnych indeksów $i$ oraz $j$ zachodzi
\[
	a_i > a_j \quad \iff \quad b_i > b_j,
\]
zaś \textit{odwrotnie monotoniczne}, gdy
\[
	a_i > a_j \quad \iff \quad b_i < b_j.
\]

\vspace{10px}

\noindent
Przykładowo ciągi $(2, 1, 3)$ oraz $(101, 100, 102)$ są jednakowo monotoniczne, zaś ciągi $(1, 2, 3)$ oraz $(10, 9, 8)$ są odwrotnie monotoniczne. Możemy teraz inaczej sformułować wykazaną wcześniej nierówność.

\vspace{10px}

\noindent
Jeśli mamy dwa ciągi $(a_1, a_2, ..., a_n)$ oraz $(b_1, b_2, ..., b_n)$ i rozpatrzymy wszystkie ich możliwe permutacje, to wartość wyrażenia
\[
	a_1b_1 + a_2b_2 + ... a_nb_n
\]
będzie największa, gdy ciągi $\{a_i\}$ oraz $\{b_i\}$ są jednakowo monotoniczne, zaś najmniejsza, gdy $\{a_i\}$ oraz $\{b_i\}$ są odwrotnie monotoniczne.

\vspace{10px}

\heading{Przykład 2}

\noindent
Liczby $a_1$, $a_2$, ..., $a_n$ są dodatnie. Wykazać, że zachodzi nierówność
\[
	a_1^3 + a_2^3 + ... + a_n^3 \geqslant a_1^2a_2 + a_2^2a_3 + ... + a^2_{n - 1}a_n + a_n^2a_1.
\]

\heading{Rozwiązanie}

\noindent
Rozpatrzmy ciągi:
\[
	(a_1^2, a_2^2, ..., a_n^2) \quad \text{oraz} \quad (a_1, a_2, ..., a_n).
\]
Zauważmy, że są one jednakowo monotoniczne, gdyż
\[
	a_i^2 > a_j^2 \iff a_i > a_j.
\]
Na mocy nierówności o ciągach jednomonotonicznych prawdą jest, że
\[
	a_1^2 \cdot a_1 + a_2^2 \cdot a_2 + ... + a_n^2 \cdot a_n \geqslant a_1^2 \cdot a_2 + a_2^2 \cdot a_3 + ... + a_n^2 \cdot a_1,
\]
co jest równoważne tezie.

\newpage

\heading{Cykliczność a symetryczność}

\noindent
Bardzo często przy rozwiązywaniu nierówności zdarzają się stwierdzenia typu ,,bez straty ogólności załóżmy, że''. Zazwyczaj chodzi o to, że rozpatrywany jest jeden przypadek, a inne są do niego analogiczne. Nie zawsze jednak są one analogiczne. Chociażby w Przykładzie 2 nie można założyć bez straty ogólności, że
\[
	a_1 \geqslant a_2  \geqslant ... \geqslant a_n,
\]
gdyż przypadek, gdy $n = 4$ i $a_1 \geqslant a_3 \geqslant a_2 \geqslant a_4$ nie daje się łatwo sprowadzić do przypadku powyższego.
\vspace{10px}

\noindent
Zdefiniujmy wyrażenie algebraiczne jako \textit{symetryczne}, gdy wyrażenie dla liczb $(a_1, a_2, ..., a_n)$ nie zmieni się, gdy rozpatrzymy dowolną ich permutację. Chociażby wyrażenie
\[
	abc + a + b + c
\]
jest symetryczne. Chociażby rozpatrując permutację $(b, a, c)$ krotki $(a, b, c)$ otrzymamy
\[
	bac + b + a + c = abc + a + b + c.
\]

\noindent
Wyrażenie algebraiczne nazwiemy \textit{cyklicznym}, gdy wyrażenie dla liczb $(a_1, a_2, ..., a_n)$ nie zmieni się, gdy rozpatrzymy dowolne ich przestawienie cykliczne -- $(a_k, a_{k + 1}, ..., a_n, a_1,  a_{k - 1})$. Wyrażenie
\[
	ab^2 + bc^2 + ca^2
\]
jest cykliczne. Dla przykładu rozpatrując przestawienie cykliczne $(c, a, b)$ krotki $(a, b, c)$ otrzymamy
\[
	ca^2 + ab^2 + bc^2 = ab^2 + bc^2 + ca^2.
\]
Nie jest ono symetryczne, gdyż dla permutacji $(b, a, c)$ krotki $(a, b, c)$ mamy
\[
	ba^2 + ac^2 + cb^2 \neq ab^2 + bc^2 + ca^2
\]

\noindent
Nierówność nazwiemy symetryczną albo cykliczną, gdy rozpatrując odpowiednio permutację lub przestawienie cykliczne zmiennych, w oczywisty sposób otrzymana nierówność jest równoważna nierówności wyjściowej.
\begin{itemize}
	\item Gdy nierówność jest symetryczna, to można założyć ustalony porządek liczb. 
	\item Jeśli zaś nierówność jest jedynie cykliczna, to można chociażby założyć, że pewna zmienna jest największa lub najmniejsza spośród wszystkich. Nie można jednak ustalać sobie ich porządku!
\end{itemize}


