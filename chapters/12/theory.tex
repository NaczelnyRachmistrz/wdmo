% Rozdział 12 – bardziej zaawansowanie nierówności

\theory{Bardziej zaawansowane nierówności}


% nierówność Cauchego–Schwarza
\heading{Nierówność Cauchego–Schwarza}

\noindent
Dla liczb rzeczywistych $a_1, a_2, ..., a_n$ oraz $b_1, b_2, ..., b_n$ zachodzi nierówność
\[
	\left(a_1^2 + a_2^2 + ... + a_n^2\right)\left(b_1^2 + b_2^2 + ... + b_n^2\right) \geqslant \left(a_1b_1 + a_2b_2 + ... + a_nb_n\right).
\]

\heading{Dowód}

% dowód z deltami jest w ,,Wędrówkach po krainie nierówności'

% ta sama w formie Engela

\heading{Nierówność Cauchego–Schwarza w formie Engela}

\noindent
Dla liczb rzeczywistych $a_1, a_2, ..., a_n$ oraz $b_1, b_2, ..., b_n$ zachodzi nierówność
\[
	\frac{a_1}{b_1} + \frac{a_2}{b_2} + ... + \frac{a_n}{b_n} \geqslant \frac{\left(a_1 + a_2 + ... + a_n\right)^2}{a_1b_1 + a_2b_2 + ... + a_nb_n}.
\]

\heading{Dowód}

% też w wędrówkach

% ciągi jednomonotoniczne

\heading{Lemat 1}

\noindent
Dane są liczby rzeczywiste $a_1 \geqslant a_2$ oraz $b_1 \geqslant b_2$. Wówczas zachodzi nierówność
\[
	a_1b_1 + a_2b_2 \geqslant a_1b_2 + a_2b_1.
\] 

\heading{Dowód}


% tutaj opis przekształceń, sortowania bąbelkowego
% ,,z tego algorytmu wynika następujące twierdzenie

\heading{Twierdzenie o ciągach jednomonotonicznych}

\noindent
Liczby $a_1 \geqslant a_2 \geqslant ... \geqslant a_n$ oraz $b_1 \geqslant b_2 \geqslant ... \geqslant b_n$ są rzeczywiste. Niech $b_1', b_2', ..., b_n'$ będzie permutacją liczb $b_1, b_2, ..., b_n$. Wówczas zachodza nierówności
\[
	a_1b_1 + a_2b_2 + ... + a_nb_n \geqslant a_1b_1' + a_2b_2' + ... + a_nb_n' \geqslant a_1b_n + a_2b_{n - 1} + ... + a_nb_1.
\]


% wniosek, że ta suma maksymalizuje się przy jednakowym uporządkowaniu – nie trzeba by a_1 > ... >a_n, tylko by ten porządek był taki jak w b_i


\heading{Przykład 1}

\noindent
Liczby $a_1$, $a_2$, ..., $a_n$ są dodatnie. Wykazać, że zachodzi nierównośc
\[
	a_1^3 + a_2^3 + ... + a_n^3 \geqslant a_1^2a_2 + a_2^2a_3 + ... + a^2_{n - 1}a_n + a_n^2a_1.
\]

\heading{Rozwiązanie}

\heading{Cykliczność a symetryczność}

% omówić częsty błąd polegający na bezpodstawnym zakładaniu monotoniczności w przykładzie pierwszym
