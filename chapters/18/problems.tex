%Source: https://sms.math.nus.edu.sg/simo/training2003/smograph.pdf P2
\begin{problem}{1}
	W pewnej grupie $2n$ osób każda osoba zna co najmniej $n$ innych osób. Wykazać, że możemy usadzić pewne cztery osoby przy okrągłym stole, aby każda z nich znała swoich sąsiadów.
\end{problem}

%Source: https://artofproblemsolving.com/community/c6h1533982p9242011
\begin{problem}{2}
	Dany jest graf, w którym każdy wierzchołek ma stopień co najwyżej $100$. Zbiór krawędzi nazwiemy \textit{idealnym}, jeśli żadne dwie krawędzie należące do niego nie mają wspólnego końca oraz nie da się dołożyć żadnej krawędzi, aby ta własność była zachowana. W każdym ruchu można usunąć dowolny zbiór idealny. Wykazać, że po wykonaniu dowolnych $198$ ruchów, graf nie będzie zawierał żadnej krawędzi.
\end{problem}

%Source: wikipedia/well-known
\begin{problem}{3}
	Dany jest graf planarny, w którym $e$ to liczba krawędzi, a $v \geqslant 3$ to liczba wierzchołków. Wykazać, że zachodzi nierówność
	\[
		e \leqslant 3v - 6.
	\]
\end{problem}

%Source: zdalne obozy naukowe
\begin{problem}{4}
	Na pewnym przyjęciu okazało się, że każdy zna co najmniej $k$ innych gości, gdzie $k \geqslant 2$ jest pewną liczbą naturalną. Wykazać, że istnieje taka liczba naturalna $n\geqslant k + 1$, że można usadzić pewnych $n$ uczestników przyjęcia przy okrągłym stole tak, aby każdy znał obu swoich sąsiadów.
\end{problem}


%Source: zdalne obozy naukowe
\begin{problem}{5}
	Krawędzi grafu pełnegot o $n \geqslant 3$ wierzchołkach zostały pokolorowane w taki sposób, że każdy kolor został użyty do pokolorowania co najwyżej $n - 2$ krawędzi. Wykazać, że istnieje podgraf tego grafu, będący trójkątem, o krawędziach parami różnych kolorów.
\end{problem}

% Source: https://dominik-burek.u.matinf.uj.edu.pl/Krynica2019.pdf P6
\begin{problem}{6}
	W pewnym państwie znajduje się 2019 miast, a wszystkie odległości między nimi są różne. Niektóre miasta są połączone lotami (w obie strony). Okazało się, że dokładnie dwa loty opuszczają każde miasto, a są to loty do dwóch najbardziej odległych miast. Udowodnij, że korzystając z lotów, z dowolnego miasta można dotrzeć do każdego innego.
\end{problem}