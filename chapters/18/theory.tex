% Rozdział 18 - Grafy 2

\theory{Grafy 2}

% Hall

\heading{Twierdzenie 1 (Halla)}

\noindent
Dana są dwa zbiory -- chłopców i dziewcząt. Niektórzy chłopcy znają się z niektórymi dziewczynami. Wówczas następujące dwa warunki są równoważne:
\begin{enumerate}
	\item każdemu chłopcowi można przyporządkować dokładnie jedną, znaną mu dziewczynę, przy czym każda dziewczyna będzie przyporządkowana do co najwyżej jednego chłopca;
	\item każdy podzbiór chłopców liczący $k$ osób zna co najmniej $k$ dziewczyn.
\end{enumerate}

% tutaj jakiś rysunek z przykładem o co cho

\heading{Dowód}

% Source: https://luckytoilet.wordpress.com/2013/12/21/halls-marriage-theorem-explained-intuitively/
\heading{Przykład 1}

\noindent
W pewnym turnieju gra $2m$ drużyn. W ciągu każdego z $2m - 1$ dni każda z drużyn rozgrywała po jednej partii. Każda drużyna zagrała z każdą drużyną dokładnie raz. W żadnym meczu nie było remisów. Wykazać, że można dla każdego dnia wybrać jedną drużynę, która w danym dniu wygrała swój mecz, tak, aby żadna drużyna nie została wybrana więcej niż raz.

\heading{Rozwiązanie}

\heading{Grafy planarne}

% definicja grafu planarnego

% rysunek

% definicja: ściany, spójne składowe

% Euler

\heading{Twierdzenie 2 (Eulera)}

\noindent
Dany jest graf planarny, w którym jest $e$ krawędzi, $v$ wierzchołków, $f$ ścian oraz $c$ spójnych składowych. Wówczas zachodzi równość
\[
	v - e + f - c = 1.
\]

\heading{Dowód}

% Teraz dwa przykłady


% Przykład z pajączkami
% source:
\heading{Przykład 2}

\noindent
Dany jest graf o $n$ wierzchołkach posiadający $k$ krawędzi o długościach
będących różnymi liczbami całkowitymi od $1$ do $k$. Udowodnić, że istnieje w
nim ścieżka składająca się z co najmniej $\frac{2k}{n}$ krawędzi, których długości tworzą ciąg rosnący.

\vspace{5px}

\heading{Rozwiązanie}

% Przykład z probabilem
% source: https://om.mimuw.edu.pl/static/app_main/camps/oboz2018.pdf P30
\heading{Przykład 3}

\noindent
Na dyskotece spotkało się $n$ chłopców i $m$ dziewczyn. Okazało się, że każda chłopiec zna co najwyżej $2018$ dziewcząt oraz każda para dziewcząt ma wspólnego znajomego chłopca. Wykazać, że da się każdej dziewczynie przypisać podzbiór znajomych jej chłopców w taki sposób, by przypisane podzbiory były parami rozłączne oraz by było co najmniej $\frac{m(m−1)}{2018}$ różnych par dziewcząt mających wspólnego znajomego chłopca przypisanego do jednej z nich.

\vspace{5px}

\heading{Rozwiązanie}