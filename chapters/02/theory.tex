% Rozdział 2 – równania funkcyjne

\theory{Równania funkcyjne}

\heading{Przykład 1}

\noindent
Znajdź wszystkie funkcje $f:\mathbb{R}\rightarrow\mathbb{R}$ spełniające dla wszystkich $x, y \in \mathbb{R}$, równanie 
\[
    f(x + y) = f(x) − f(y).
\]

\heading{Rozwiązanie}

\noindent
Zauważmy, że skoro dane równanie jest spełnione dla wszystkich liczb rzeczywistych $x$~i~$y$ to jest spełnione w szczególności dla $x = y = 0$. Wówczas
\begin{gather*}
    f(0) = f(0) - f(0) = 0.
\end{gather*}

\noindent
Podstawiając do wyjściowej równości $x = 0$ otrzymujemy
\[
    f(y) = f(0) - f(y).
\]
Na mocy wyżej wykazanej zależności $f(y) = 0$ mamy
\begin{gather*}
    f(y) = - f(y) \\
    f(y) = 0.
\end{gather*}
Wykazaliśmy, że $f(x) = 0$ dla wszystkich liczb rzeczywistych $x$. Pozostaje sprawdzić, że istotnie taka funkcja spełnia warunki zadania. Zauważmy, że wówczas
\[
    f(x + y) = 0 = f(x) - f(y).
\]

\qed

\vspace{10px}

\noindent
Metodę, polegająca na podstawianiu szczególnych wartości do danego równania, jest najważniejszym narzędziem w walce z równaniami funkcyjnymi. Często, aby zadania rozwiązać, należy użyć jej kilka lub nawet kilkanaście razy.

\vspace{10px}

\noindent
Należy zaznaczyć, że bardzo często rozwiązując równanie funkcyjne, wyznacza się zbiór funkcji, które mogą spełniać dane równanie. Jednak często nie oznacza to, że muszą one go spełniać, gdyż podstawianie zazwyczaj nie jest przejściem równoważnym. Dlatego należy zawsze w swoim rozwiązaniu zawrzeć sprawdzenie tego, czy otrzymane funkcje istotnie działają. Brak takiego sprawdzenie w większości przypadków skutkuje obniżeniem oceny za dane zadanie.

\vspace{20px}

\heading{Przykład 2}

\noindent
Znajdź wszystkie funkcje $f:\mathbb{R}\rightarrow\mathbb{R}$ spełniające dla wszystkich $x, y \in \mathbb{R}$ równanie 
\[
    f(2f(x) + f(y)) = 2x + f(y).
\]

\newpage

\heading{Rozwiązanie}

\noindent
Rozwiązanie rozpoczniemy od wykazania następującego lematu.

\vspace{10px}

\noindent
\underline{Lemat 1.} Dla każdego $a \in \mathbb{R}$ istnieje $x \in \mathbb{R}$, że $f(x) = a$.

\vspace{5px}

\noindent
Podstawmy $\dfrac{a - f(y)}{2}$ w miejsce zmiennej $x$
\[
    f\left(f\left(\frac{a - f(y)}{2}\right) + f(y)\right) = 2\left(\frac{a - f(y)}{2}\right) + f(y) = a.
\]

\noindent
Zauważmy, że z otrzymanej równości wynika teza lematu – liczbę $a$ można wybrać dowolnie, zaś po prawej stronie otrzymamy argument, dla którego funkcja przyjmie tę wartość.

\vspace{10px}

\noindent
Korzystając z lematu, podstawmy w miejsce $y$ taką liczbę $a$, aby $f(a) = -2f(x)$. Wówczas
\begin{gather*}
    f(2f(x) + f(a)) = 2x + f(a) \\
     f(0) = 2x - 2f(x) \\
    f(x) = x  + \frac{1}{2}f(0).
\end{gather*}
Podstawiając do powyższej równości $x = 0$ otrzymujemy, że $f(0) = 0$. Stąd
\[
    f(x) = x + \frac{1}{2}f(0) = x.
\]
Sprawdzamy, że funkcja $f(x) = x$ istotnie spełnia warunki zadania.

\qed

\vspace{10px}

\noindent
W powyższym rozumowaniu kluczowe było wykazanie, że dana funkcja przyjmuje wszystkie wartości rzeczywiste -- inaczej mówiąc jest surjekcją. Mogliśmy także wykazać więcej, mianowicie, że dana funkcja jest różnowartościowa. Zakładając, że $f(a) = f(b)$ dla pewnych liczb $a,\;b$ podstawiamy w miejsce $(x, y)$ kolejno $(a, 0)$ i $(b, 0)$ otrzymując
\[
    f(2f(a) + f(y)) = 2a + f(y) \quad \text{oraz} \quad f(2f(b) + f(y)) = 2b + f(y).
\]
Na mocy wyżej założonej równości lewe strony obu zależności są sobie równe. Stąd prawe również, skąd $a = b$. Implikacja $f(a) = f(b) \implies a = b$ jest równoważna temu, że funkcja~$f$ jest różnowartościowa.

\vspace{10px}

\noindent
W większości rozwiązań funkcyjnych konieczne będzie wykonanie wielu ,,sztampowych'' podstawień i spróbować wykazać własności funkcji -- chociażby te wspomniane wyżej. Niekiedy do rozwiązania zadania potrzebny będzie błyskotliwy pomysł czy nietypowe połączenie faktów. W innych zaś przypadkach samo rzetelne i uważne próbowanie znanych trików może okazać się wystarczające. 

\vspace{10px}
