\newpage
\solutions{Równania funkcyjne}

\begin{problem}{1} 
	Znajdź wszystkie funkcje $\mathbb{R} \rightarrow \mathbb{R} $ spełniające dla wszystkich $x, y \in \mathbb{R} $ równanie $f(x)+f(y) = f(xy) $.
\end{problem}

\answer{Jedyną funkcją spełniającą warunki zadania jest $f(x) = 0$.}

\noindent
Podstawmy $y=0$: \[ f(x) + f(0) = f(0), \] czyli $f(x) =0$ dla każdego x. Łatwo sprawdzić, że ta funkcja spełnia warunki zadania. 

\vspace{10px}

\begin{problem}{2}
	Znajdź wszystkie funkcje $\mathbb{R} \rightarrow \mathbb{R} $ spełniające dla wszystkich $x, y \in \mathbb{R} $ równanie $f(x-f(y)) = 1 - x - y$.
\end{problem}

\noindent
Podstawmy $x = f(y)$. Otrzymamy 
\begin{gather*}
	f(0) = 1 - f(y) - y, \\
	f(y) = - y + (1 - f(0)).
\end{gather*} 
Podstawmy $y = 0$ do powyższej zależności. Wówczas łatwo obliczyć, że $f(0)=\frac{1}{2}$. Czyli $f(x) = - x + \frac{1}{2} $. Ta funkcja istotnie spełnia warunki zadania, gdyż
\[
	f(x - f(y)) = f(y) - x + \frac{1}{2} = 1 - y - x .
\]

\begin{problem}{3}
	Znajdź wszystkie funkcje $\mathbb{R} \rightarrow \mathbb{R} $ spełniające dla wszystkich $x, y \in \mathbb{R} $ równanie $f(x^{2}y) = f(xy) + yf(f(x) +y) $.
\end{problem}

\answer{Funkcja $f(x) = x$ jest jedynym rozwiązaniem.}

\noindent
Podstawmy $x = 0$ i $y = -f(0)$. Otrzymamy $f(0)^{2}=0$, czyli $f(0)=0$. Podstawmy $x=0$: 
\[ 
	0 = yf(f(0) + y) = yf(y)
\] 
Dla niezerowego $y$ mamy $f(y) = 0$.
Sprawdzamy, że funkcja $f(x) = 0$ spełnia warunki zadania. Łącząc powyższe wnioski otrzymujemy, że jedyni funkcja $f(x)=0$ spełnia warunki zadania. 

\newpage

\begin{problem}{4}
	Znajdź wszystkie funkcje $\mathbb{R} \rightarrow \mathbb{R} $ spełniające dla wszystkich $x, y \in \mathbb{R} $ równanie $2f(x)+f(1-x)=x^{2} $. 
\end{problem}

\answer{Jedyną funkcją spełniającą warunki zadania jest $f(x) = \frac{2x^2 - (1-x)^2}{3}$.}

\noindent
Podstawmy $1-x$ za x. Otrzymamy 
\[ 
	2f(1 - x) + f(x) = 2(1 - x)^{2}.
\] 
Z równaniem z zadania tworzy to układ równań ze zmiennymi $f(x)$ i $f(1 - x)$. Wyliczamy $f(x) = \frac{2x^{2}-(1-x)^{2}} {3} $. Wystarczy teraz tylko sprawdzić, że ta funkcja spełnia warunki zadania. 

\vspace{5px}

\begin{problem}{5}
	Znajdź wszystkie funkcje $\mathbb{R} \rightarrow \mathbb{R} $ spełniające dla wszystkich $x, y \in \mathbb{R} $ równanie $f(x+y) = f(f(x)) + y + 1$. 
\end{problem}

\answer{$f(x) = x - 1$ jest jedynym rozwiązaniem danego równania.}

\noindent
Podstawmy $x = 0$: 
\[
 f(y) = f(f(0)) + 1 + y, 
\] 
czyli $f(x) = x + a$ dla pewnego stałego a. Podstawmy tę funkcję do wyjściowego równania
\[
	x+y+a=x+2a+y+1.
\] 
Mamy $a = -1$. Łatwo sprawdzić, że funkcja $f(x) = x - 1$ spełnia warunki zadania. 

\vspace{5px}

\begin{problem}{6}
	Znajdź wszystkie funkcje różnowartościowe
	$\mathbb{R} \rightarrow \mathbb{R} $ spełniające dla wszystkich $x, y \in \mathbb{R} $ równość $f(f(x) + y)  = f(x+y) + 1$.
\end{problem}

\answer{Jedyną funkcją spełniającą warunki zadania jest $f(x) = x + 1$.}

\noindent
Podstawmy $y = -x$. Otrzymamy 
\[ 
	f(f(x) - x) = f(0) + 1.
\] 
Zauważmy, że prawa strona równości jest stała. Z różnowartościowości f wynika, że wartość $f(x) - x$ jest stała. Czyli $f(x) - x = a$ dla pewnego a. Wstawiamy $f(x) = x + a$ do wyjściowego równania 
\[
	x + y + 2a = x + y + a + 1,
\]
więc $a = 1$. Skąd $f(x) = x + 1 $ -- możemy sprawdzić, że ta funkcja spełnia warunki zadania.

\newpage

\begin{problem}{7}
	Znajdź wszystkie funkcje 
	$\mathbb{R} \rightarrow \mathbb{R} $ spełniające dla wszystkich $x, y \in \mathbb{R} $ nierówność $f(x^2+y) + f(y) \geqslant f(x^2) + f(x) $. 
\end{problem}

\noindent
Podstawmy $x = 0$
\[ 
	f(y) \geqslant f(0). 
\] 
Podstawmy $y = 0$ 
\[ 
	f(x^2) + f(0) \geqslant f(x^2) + f(x), 
\] 
czyli $f(0) \geqslant f(x) $. 
Łącząc oba wnioski otrzymuję 
\[ 
	f(0) \geqslant f(x) \geqslant f(0), 
\] 
czyli $f(x) = f(0)$. Innymi słowy $f$ jest funkcją stałą. Łatwo zauważyć, że taka funkcja spełnia warunki zadania. 

\vspace{5px}

\begin{problem}{8}
	Znajdź wszystkie funkcje $\mathbb{Z} \rightarrow \mathbb{Z} $ spełniające dla wszystkich $x \in \mathbb{Z} $ równanie $f(f(x)) =x+1$. 
\end{problem}

\answer{Szukane funkcje nie istnieją.}

\noindent
Zauważmy, że zachodzą równości 
\begin{gather*} 
f(f(f(x))) = f(x + 1) \\
f(f(f(x))) = f(x) +1 
\end{gather*} 
Z tego otrzymujemy równość: 
\[
	f(x) =f(x-1)+1 
\] 
Skoro działamy w liczbach całkowitych to możemy wywnioskować, że
\[
	f(x) = f(x - 1) + 1 = f(x - 2) +  2 =  ... = x + f(0).
\] 
Podstawmy równość $f(x) = x + f(0)$ do 
$f(f(x)) = x + 1$:
\[
	x + 1 = f(f(x))= x + 2f(0), 
\]
czyli $f(0) = \frac{1}{2}$. Sprzeczność. Takie funkcje nie istnieją.

\vspace{5px} 

\begin{problem}{9}
	Znajdź wszystkie funkcje $\mathbb{R} \rightarrow \mathbb{R} $ spełniające dla wszystkich $x, y \in \mathbb{R} $ równanie $f(x) f(y) = f(x-y) $. 
\end{problem}

\answer{Daną zależność spełniają funkcje $f(x) = 1$ i $f(x) = 0$.}

\noindent
Podstawmy $x = y = 0$. Wtedy otrzymujemy 
\[
	f(0)^{2} = f(0) \implies f(0) \in \{0, 1\}.
\]
Podstawmy $x = y$
\[
	f(x)^{2} = f(0).
\] 
Jeśli $f(0)=0$, to $f(x) =0$. Łatwo sprawdzić, że funkcja zerowa spełnia warunki zadania.
Zobaczmy, co jeśli $x = y$ oraz $f(0) = 1$:
\[
	f(x)^{2} = f(0) = 1 
\] 
Czyli $f(x)$ jest równe $-1$ lub 1 dla każdego $x$.
Podstawmy $x=0$
\[
	f(y) =f(-y).
\] 
Zauważmy, że 
\[
	f(x - y) = f(x)f(y) = f(x)f(-y) = f(x+y). 
\] 
Weźmy 2 dowolne liczby a i b. Biorąc $x=\frac{a+b}{2}$ oraz 
$y=\frac{a-b}{2}$ otrzymamy
\[
	f(x+y) = f(x-y) \implies f(a) = f(b).
\] 
Skoro $a$ i $b$ były dowolne to $f$ jest funkcją stałą, czyli $f(x) = 1$. Łatwo sprawdzić, że ta funkcja również spełnia warunki zadania. Czyli tę zależność spełniają funkcje $f(x) = 1 $ i $f(x) = 0$. Sprawdzamy, że istotnie one działają.

\vspace{5px}

\begin{problem}{10}
	Udowodnij, że nie istnieje taka funkcja $f:\mathbb{R}\longrightarrow\mathbb{R}$, że dla dowolnych liczb rzeczywistych $x$, $y$ zachodzi równość:
	\[
	f(f(x)+2f(y))=x+y.
	\]
\end{problem}

\noindent
\underline{Lemat 1} Funkcja $f$ jest różnowartościowa

Załóżmy, że $f(a) = f(b)$. Podstawmy, $x=a$ oraz $x = b$
\[
	f(f(a) + 2f(y)) = a + y \quad \text{oraz} \quad f(f(b) + 2f(y)) = b + y.
\] 
Skoro $f(a)=f(b)$, to 
\[
	f(f(a) + 2f(y)) = f(f(b) + 2f(y)),
\] 
a więc $a + y = b + y$, czyli $a = b$. A więc $f$ istotnie jest różnowartościowa.
\vspace{10px}

\noindent
Zauważamy, że zachodzą równości
\[
	f(f(x) + 2f(y)) = x + y \quad \text{oraz} \quad f(f(y)+2f(x))=x+y.
\]
Czyli 
\[
	f(f(x) + 2f(y)) = f(f(y) + 2f(x)).
\] 
Skoro $f$ jest różnowartościowa, to 
\[
	f(x)+2f(y)=f(y)+2f(x),
\]
więc $f(x)=f(y)$ dla wszystkich liczb $x$, $y$. Czyli $f$ musiałaby być funkcją stałą, a to jest oczywista sprzeczność z danym równaniem.
