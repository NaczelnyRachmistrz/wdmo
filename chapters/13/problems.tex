% https://www.isinj.com/mt-usamo/Functional%20Equations%20-%20Marko%20Radovanovic.pdf P3
\begin{problem}{1}
	Dana jest funkcja spełniająca dla dowolnej liczby rzeczywistej $x$ równość
	\[
		f(f(x)) - f(x) = x.
	\]
	Znaleźć liczbę liczb rzeczywistych $a$, takich, że $f(f(a)) = 0$.
\end{problem}

% https://artofproblemsolving.com/community/c6h319947
\begin{problem}{2}
	Znajdź wszystkie funkcje  $ f: \mathbb{Z}_{>0} \to \mathbb{Z}_{>0}$ takie, że  $f(f(n)) + (f(n))^2 = n^2 + 3n + 3$.
\end{problem}


% https://web.evanchen.cc/handouts/FuncEq-Intro/FuncEq-Intro.pdf
\begin{problem}{3}
	Znajdź wszystkie funkcje $f:\mathbb{R}\longrightarrow\mathbb{R}$, że dla dowolnych liczb rzeczywistych $x$, $y$ zachodzi równość:
	\[
		f(x^2 + y) = f(x^{27} + 2y) + f(x^4).
	\]
\end{problem}

%Source: https://artofproblemsolving.com/community/c6h1414672p7967490 Kanada 2017 P2
\begin{problem}{4}
	Niech $f(n)$ będzie funckją z dodatnich liczb całkowitych w dodatnie liczby całkowite. Wiadomo, że dla każdej dodatniej liczby całkowitej $n$ liczba  $f(f(n))$ jest liczbą dodatnich dzielników $n$. Wykazać, że jeśli $p$ jest liczbą pierwszą, wówczas $f(p)$ również jest liczbą pierwszą.
\end{problem}

\begin{problem}{5}
	Znajdź wszystkie funkcje $f:\mathbb{R}\longrightarrow\mathbb{R}$, że dla dowolnych liczb rzeczywistych $x$, $y$ zachodzi równość:
	\[
		f(x^2+f(y))=y+f(x)^2.
	\]
\end{problem}

\begin{problem}{6}
	Wykazać, że istnieje taka funckja $f:\mathbb{R}\longrightarrow\mathbb{R}$, że nie istnieje taka funckja $g:\mathbb{R}\longrightarrow\mathbb{R}$, że zachodzi równość $f(x) = g(g(x))$.
\end{problem}

