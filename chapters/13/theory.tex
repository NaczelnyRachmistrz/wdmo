% Rozdział 12 – Równania funkcyjne i wielomianowe

\theory{Równania funkcyjne 2}

% Funckje spełniające równanie cauchego

\heading{Przykład 1}

\noindent
Wyznaczyć wszytskie funkcje $f:\mathbb{Q}\longrightarrow\mathbb{Q}$, które dla dowolnych $x, y \in \mathbb{Q}$ spełniają warunek
\[
	f(x) + f(y) = f(x + y).
\]

\heading{Rozwiązanie}



% Dowód własności z monotonicznością

\heading{Twierdzenie 1}

\noindent
Jeśli pewna funkcja $f:\mathbb{R}\longrightarrow\mathbb{R}$ dla dowolnych liczb rzeczywistych $x, y$ spełnia warunek

\[
	f(x) + f(y) = f(x + y).
\]
oraz jest funkcja rosnącą, to wówczas $f(x) = ax$.

\heading{Dowód}


\heading{Przykład 2}
% przykład – zadanie z IMO 2002
% https://artofproblemsolving.com/community/c6h17333p118703

\noindent
Znajdź wszystkie funkcje rzeczywiste $f$, że 
\[ 
	\left(f(x)+f(z)\right)\left(f(y)+f(t)\right)=f(xy-zt)+f(xt+yz)  
\] 
dla wszystkich liczb rzeczywistych $x,y,z,t$.

\heading{Rozwiązanie}

% komentarz – warto robić małe kroki



% https://web.evanchen.cc/handouts/FuncEq-Intro/FuncEq-Intro.pdf Ex 2.1
\heading{Przykład 3}

\noindent
Znajdź wszystkie funkcje $ f:\mathbb{R}\to\mathbb{R} $ dla których 
\[
	f(f(x)^2+f(y)) = xf(x)+y 
\],
dla dowolnych liczb rzeczywistych $x, y$.

\heading{Rozwiązanie}
