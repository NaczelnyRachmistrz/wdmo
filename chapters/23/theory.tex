% Rozdział 23 – Geometria kombinatoryczna

\theory{Geometria kombinatoryczna}

% skrypt w większości zadaniowy zadaniowy

% https://om.mimuw.edu.pl/static/app_main/camps/oboz2019.pdf P35
\heading{Przykład 1}

\noindent
Na płaszyźnie danych jest $n$ parami różnych okręgów, przy czym:
\begin{itemize}
	\item każdy z nich ma promień 1,
	\item żadne dwa z nich nie są styczne,
	\item nie ma wśród nich okręgu rozłącznego z pozostałymi okręgami.
\end{itemize}

\noindent
Niech $A$ będzie zbiorem wszystkich punktów przecięcia danych okręgów. Wykazać, że $|A| \geqslant n$.

\heading{Rozwiązanie}

% powiedzieć, że trzeba skorzystać w sposób istotny z geometrii

\heading{Przykład 2}
% tw. Sylvestera-Gallai

\noindent
Danych jest $n$ punktów na płaszczyźnie, przy czym nie istnieje prosta, która zawiera je wszystkie. Wykazać, że istnieje prosta, która zawiera dokładnie dwa spośród tych $n$ punktów.

\heading{Rozwiązanie}
