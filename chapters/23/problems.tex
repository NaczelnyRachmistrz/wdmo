\begin{problem}{1}
	Każdemu punktowi $A$ na płaszczyźnie przyporządkowano pewną liczbę rzeczywistą $f(A).$ Wiadomo, że $f(M)=f(A)+f(B)+f(C),$ gdy $M$ jest środkiem masy trójkąta $ABC.$ Wykazać, że $f(A)=0$ dla wszystkich punktów $A.$
\end{problem}

\begin{problem}{2}
	Dana jest liczba $n \geqslant 3$ oraz pewne $n$ par liczb rzeczywistych  $(x_1,\; y_1)$, $(x_2,\; y_2)$, ..., $(x_n, \; y_n)$, że dla każdej liczby całkowitej $1 \leqslant i \leqslant n$ zachodzi
	\[
		x_i^2 + y_i^2 = 1.
	\]
	Wykazać, że istnieją takie liczby $i$ oraz $j$, że zachodzi nierówność
	\[
		(x_i - x_{j})^2 + (y_i - y_{j})^2 \leqslant \frac{4\pi^2}{n^2},
	\]
	przy czym $x_{n + 1} = x_1$ oraz $y_{n + 1} = y_1$.
\end{problem}


\begin{problem}{3}
	Rozstrzygnąć, czy istnieje taki wielokąt wypukły, że dla każdego jego boku i każdego jego wierzchołka, przy czym rozpatrywany wierzchołek nie należy do rozpatrywanego boku, rzut prostokątny tego wierzchołka na prostą zawierającą ten bok leży poza rozpatrywanym bokiem.
\end{problem}

\begin{problem}{4}
	Danych jest 100 czerwonych punktów na płaszczyźnie, przy czym żadne trzy nie leżą na jednej prostej. Prostą \textit{dzielącą} będziemy nazywać każdą prostą, która przechodzi przez dwa czerwone punkty oraz po obu jej stronach jest po 49 czerwonych punktów. Wykaż, że istnieje przynajmniej 50 prostych \textit{dzielących}.
\end{problem}
 
\begin{problem}{5}
	W pewnym $n$-kącie foremnym poprowadzono $n - 3$ przekątne, tak, że żadne dwie z nich się nie przecinają w punkcie innym niż wierzchołek danego wielokąta oraz z każdego wierzchołka wychodzi nieparzysta liczba przekątnych. Wykazać, że liczba $n$ jest podzielna przez $3$.
\end{problem}

\begin{problem}{6}
	Znaleźć wszystkie skończone zbiory $\mathcal{S}$ punktów na płaszczyźnie, że dla dowolnych trzech punktów $A$, $B$, $C \in \mathcal{S}$ istnieje taki punkt $D \in \mathcal{S}$, że czworokąt $ABCD$ jest równoległobokiem.
\end{problem}