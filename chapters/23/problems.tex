%Source: https://artofproblemsolving.com/community/c6h1826975p12222051
\begin{problem}{1}
	Każdemu punktowi $A$ na płaszczyźnie przyporządkowano pewną liczbę rzeczywistą $f(A).$ Wiadomo, że $f(M)=f(A)+f(B)+f(C),$ gdy $M$ jest środkiem masy trójkąta $ABC.$ Wykazać, że $f(A)=0$ dla wszystkich punktów $A.$
\end{problem}

%Source: https://alexanderrem.weebly.com/uploads/7/2/5/6/72566533/miscellaneousproblemssolutions.pdf
\begin{problem}{2}
	Punktemem kratowymym nazwiemy punkt o obu współrzędnych całkowitych. Cyrkiel wbito w taki sposób, że dwa jego końce znajdują się w punktach kratowych. Odległości między punktami, w które cyrkiel jest wbity, nie da się zmienić. Jest możliwe, aby nie zmieniając punktu wbicia jednej nogi cyrkla, zmienić położenie wbicia drugiej -- oczywiście zachowując odległość między tymi punktami. Rozstrzygnąć, czy istnieją takie dwa punkty początkowe, że wykonując pewną liczbę skończonych operacji, da się zamienić nogi cyrkla miejscami.
\end{problem}

%Source: 
\begin{problem}{3}
	Rozstrzygnąć, czy istnieje taki wielokąt wypukły, że dla każdego jego boku i każdego jego wierzchołka, przy czym ropzatrywany wierzchołek nie należy do rozpatrywanego boku, rzut prostokątny tego wierzchołka na prostą zawierającą ten bok leży poza rozpatrywanym bokiem.
\end{problem}

%Source: 
\begin{problem}{4}
	Danych jest 100 czerwonych punktów na płaszczyźnie, przy czym żadne trzy nie leżą na jednej prostej. Prostą \textit{dzielącą} będziemy nazywać każdą prostą, która przechodzi przez dwa czerwone punkty oraz po obu jej stronach jest po 49 czerwonych punktów. Wykaż, że istnieje przynajmniej 50 prostych \textit{dzielących}.
\end{problem}

 
\begin{problem}{5}
	W pewnym $n$-kącie foremnym poprowadzono $n - 3$ przekątne, tak, że żadne dwie z nich się nie przecinają w punkcie innym niż wierzchołek danego wielokąta oraz z każdego wierzchołka wychodzi nieparzysta liczba przekątnych. Wykazać, że liczba $n$ jest podzielna przez $3$.
\end{problem}


%Source: https://alexanderrem.weebly.com/uploads/7/2/5/6/72566533/miscellaneousproblems.pdf P3
\begin{problem}{6}
	Znaleźć wszystkie skończone zbiory $\mathcal{S}$ punktów na płaszczyźnie, że dla dowolnych trzech punktów $A$, $B$, $C \in \mathcal{S}$ istnieje taki punkt $D \in \mathcal{S}$, że czworokąt $ABCD$ jest równoległobokiem.
\end{problem}