\hints{Kombinatoryka z elementami geometrii}

\begin{hints_list}
	\item Rozpatrz nastepujące punkty\\
	\begin{center}
	\begin{tikzpicture}
		\tkzDefPoint(0,0){A}
		\tkzDefPoint(2,0){B}
		\tkzDefTriangle[equilateral](A,B) \tkzGetPoint{C}
		\tkzDefMidPoint(A,B) \tkzGetPoint{M}
		\tkzDefMidPoint(B,C) \tkzGetPoint{K}
		\tkzDefMidPoint(A,C) \tkzGetPoint{L}

		\tkzCentroid(A,B,C) \tkzGetPoint{O}
		\tkzCentroid(A,L,M) \tkzGetPoint{O_1}
		\tkzCentroid(B,K,M) \tkzGetPoint{O_2}
		\tkzCentroid(C,K,L) \tkzGetPoint{O_3}

		\tkzDrawPoints(A,B,C,K,L,M,O,O_1,O_2,O_3)
		\tkzDrawSegments(A,B B,C C,A K,L L,M M,K)
		\tkzLabelPoints[below](A,M,B)
		\tkzLabelPoints[right](K)
		\tkzLabelPoints[above](C)
		\tkzLabelPoints[left](L)
	\end{tikzpicture}
	\end{center}
	\item Zauważ, że te punkty leżą na jednym okręgu oraz trzeba pokazać, że odległość między pewnymi dwoma kolejnymi jest ogarniczona.
	\item Rozpatrz najdłuższy bok danego wielokąta.
	\item Weź dowolny punkt i dowolną prostą. Zacznij ją obracać, aż obróci się o $180\degree$. Wykaż, że w pewnym momencie była ona prostą dzielącą.
	\item Zauważ, że wszystkie trójkąty przybrzeżne są tego samego koloru -- przyjmijmy bez straty ogólności, że są czarne.
	\item 
\end{hints_list}