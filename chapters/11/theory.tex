% Rozdział 11 – gry

\theory{Gry}

\noindent
Zajmiemy się teraz analizą gier. Większość gier, które będziemy rozważać, ma pewne cechy wspólne:
\begin{itemize}
	\item dwóch graczy wykonuje ruchy na przemian,
	\item obaj gracze mają wszystkie informacje na temat stanu gry -- nie ma żadnych elementów tajnych,
	\item gra musi się skończyć -- nie można grać w nieskończoność.
\end{itemize}
Oczywiście niektóre zadania wyłamią się z tych zasad.

\vspace{10px}
\noindent
Powiemy, że jeden z graczy ma \textit{strategię wygrywającą}, gdy może tak wykonywać ruchy, aby wygrać, niezależnie od tego jak gra przeciwnik. Przykład takiej strategii ilustruje następujące zadanie.

\vspace{10px}


\heading{Przykład 1}

\noindent
Ania i Bartek grają w następującą grę. Mają do dyspozycji stół w kształcie koła oraz dowolnie wiele monet o jednakowej średnicy, mniejszej od średnicy stołu. W każdym ruchu gracz kładzie monetę na stole tak, aby nie przykrywała żadnej innej monety. Gdy wykonanie ruchu nie jest możliwe, gracz którego jest kolej przegrywa. Rozstrzygnąć który z graczy ma strategię wygrywającą, jeśli Ania wykonuje pierwszy ruch.

\vspace{5px}

\heading{Rozwiązanie}

\noindent
Wykażemy, że Ania -- która rozpoczyna grę, ma strategię wygrywającą. Składa się ona z dwóch kroków.

\begin{itemize}
	\item W pierwszym ruchu Ania kładzie monetę tak, aby środek stołu pokrywał się ze środkiem monety. Może tak zrobić, gdyż średnica stołu jest większa niż średnica monety.
	\item W kolejnych ruchach Ania kładzie monetę tak, aby była ona symetryczna do monety położonej poprzednio przez Bartka względem środka stołu.
\end{itemize}

\begin{center}
	\begin{tikzpicture}[scale = 0.4]
	    \tkzDefPoint(0,0){O}
	    \tkzDefPoint(4,0){X}
	    \tkzDrawCircle(O,X)
	    \tkzDefPoint(0,0){O_1}
	    \tkzDefPoint(1,0){X_1}
	    \tkzDrawCircle(O_1,X_1)

	    \tkzDrawPoint(O)

	    % 1st move

	    \tkzDefPoint(2.3,1.3){OA_1}
	    \tkzDefPoint(2.3,2.3){A_1}
	    \tkzDrawCircle(OA_1,A_1)
	    \tkzDefPoint(-2.3,-1.3){O_1}
	    \tkzDefPoint(-2.3,-2.3){A_1}
	    \tkzDrawCircle[dashed](O_1,A_1)

	    % 2nd move

	    \tkzDefPoint(-1.2,2.6){OA_1}
	    \tkzDefPoint(-1.2,3.6){A_1}
	    \tkzDrawCircle(OA_1,A_1)
	    \tkzDefPoint(1.2,-2.6){O_1}
	    \tkzDefPoint(1.2,-3.6){A_1}
	    \tkzDrawCircle[dashed](O_1,A_1)
	\end{tikzpicture}

	\vspace{10px}

	\begin{minipage}{0.6\textwidth}
	Przykład gry. Okręgi z linii ciągłej zostały położone przez Anię, a te z linii przerywanej przez Bartka.
	\end{minipage}
\end{center}

\noindent
Pozostaje tylko zauważyć, że po ruchu Ani sytuacja na stole jest symetryczna względem środka. Jeśli więc Bartek jest w stanie w pewnym miejscu ułożyć monetę, to oznacza, że jest to możliwe również w miejscu do niego symetrycznym. Łatwo zauważyć, że skoro moneta Bartka nie może zostać położona na środku stołu -- bo już tam leży moneta Ani -- to ruch Bartka nie wpłynie na możliwość położenia monety w miejscu symetrycznym. Stąd też jeśli Bartek jest w stanie wykonać ruch, to jest go w stanie później wykonać Ania. Dowodzi to faktu, że strategia Ani jest wygrywająca.

\qed

\vspace{10px}

\noindent
Warto zauważyć, że Bartek w drugim ruchu nie jest w stanie położyć monety symetrycznej do monety Ani, gdyż te monety się pokrywają. Pomysłowe ułożenie jej na środku pozwala jednak zastosować Ani pomysł z symetrią. Zapobiega ona w ten sposób sytuacji, w której dwie symetryczne do siebie monety będą na siebie nachodziły.

\vspace{10px}

\noindent
Motyw gry symetrycznej jest bardzo częsty w grach, stąd też zawsze należy go rozważyć. Warto także spróbować gry w szczególnych przypadkach, gdyż one często pozwalają nam zgadnąć kto ma strategię wygrywającą i nasuwają pomysł na nią. Przykładowo, jeśli dla powyższego zadania rozważymy stół o bardzo niewiele większy niż jedna moneta, to po pierwszym ruchu Ani, Bartek nie będzie w stanie położyć kolejnej monety. Nasuwa to pomysł, że Ania ma strategię wygrywającą.

\vspace{10px}

\heading{Przykład 2}

\noindent
Na stole leży $10000$ kamieni. Ruch polega na zabraniu ze stołu $2$ lub $5$ kamieni. Dwaj gracze wykonują ruch na przemian, gdy gracz nie może wykonać ruchu przegrywa. Rozstrzygnąć, który z graczy – pierwszy czy drugi – ma strategię wygrywającą.

\vspace{5px}

\heading{Rozwiązanie}

\noindent
Wprowadźmy dwa pojęcia. Jeśli na stole leży $k$ kamieni i gracz wykonujący ruch może grać w taki sposób, aby wygrać niezależnie od ruchów przeciwnika, to powiemy, że $k$ kamieni na stole jest \textit{stanem wygrywającym}. Gdy zaś to gracz, który aktualnie nie wykonuje ruchu ma strategię wygrywającą, to ten stan nazwiemy \textit{stanem przegrywającym}.

\vspace{10px}
\noindent
Zauważmy, że $k = 2$ i $k = 5$ są stanami wygrywającymi, gdyż gracz może wziąć wszystkie kamienie, co spowoduje brak możliwości ruchu następnego gracza. Zaś $k = 1$ jest stanem przegrywającym, gdyż gracz, który ma wykonać ruch, nie jest w stanie tego zrobić.


\vspace{10px}
\noindent
Do rozwiązania zadania przydadzą się dwie ważne obserwacje:
\begin{itemize}
	\item stan jest wygrywający wtedy i tylko wtedy, gdy wskutek ruchu, można przejść do pewnego stanu przegrywającego; 
	\item stan jest przegrywający wtedy i tylko wtedy, gdy wskutek dowolnego ruchu gracz go wykonujący przechodzi do stanu wygrywającego.
\end{itemize}
Formalne uzasadnienie powyższych zdań pozostawiamy czytelniczce/czytelnikowi jako ćwiczenie. Jednak są to fakty dość intuicyjne.


\vspace{10px}
\noindent
Zauważmy, że ze stanu $k = 3$ można przejść do stanu $k = 1$, który jest przegrywający. Toteż $k = 3$ jest stanem wygrywającym. Dla $k = 4$ jedynym możliwym ruchem jest zabranie dwóch kamieni -- doprowadza nas to do stanu wygrywającego $k = 2$ -- stąd $k = 4$ jest przegrywający. Dla $k = 6$ zabranie pięciu kamieni doprowadza do stanu przegrywającego, toteż $k = 6$ jest wygrywający. Ciekawa sytuacja ma miejsce dla $k = 7$. Wówczas oba ruchy doprowadzają do stanów wygrywających -- $k = 5$ i $k = 2$, więc $k = 7$ jest stanem przegrywającym.


\vspace{10px}
\noindent
Rozumując analogicznie możemy stworzyć tabelkę.
\begin{center}
\begin{tabular}{ |c|c|} 
 \hline
 k & Stan \\ 
 \hline
 1 & P \\ 
 \hline
 2 & W \\ 
 \hline
 3 & W \\ 
 \hline
 4 & P \\ 
 \hline
 5 & W \\ 
 \hline
 6 & W \\ 
 \hline
 7 & P \\ 
 \hline
 8 & P \\ 
 \hline
\end{tabular}
\hspace{30px}
\begin{tabular}{ |c|c|} 
 \hline
 k & Stan \\ 
 \hline
 9 & W \\ 
 \hline
 10 & W \\ 
 \hline
 11 & P \\
 \hline
 12 & W \\ 
 \hline
 13 & W \\ 
 \hline
 14 & P \\ 
 \hline
 15 & P \\ 
 \hline
 16 & W \\ 
 \hline
\end{tabular}
\end{center}
Można więc postawić hipotezę, że jeśli $k \equiv 0, 1, 4 \pmod{7}$ to stan jest przegrywający, zaś w przeciwnym wypadku jest on wygrywający. Wykażemy, że jest to prawda za pomocą indukcji matematycznej. Za bazę indukcji posłuży nam rozumowanie przeprowadzone powyżej.

\vspace{10px}
\noindent
Rozpatrzmy pewne $k \geqslant 8$.
\begin{itemize}
	\item Jeśli $k \equiv 0, 1, 4 \pmod{7}$, to łatwo sprawdzić, że z założenia indukcyjnego stany $k - 2$ i $k - 5$ w każdym z tych przypadków są wygrywające. Toteż rozpatrywane stany będą przegrywające.
	\item Dla $k \equiv 2, 3, 6 \pmod{7}$ zabranie dwóch kamieni doprowadza do stanu przegrywającego, toteż stan $k$ jest wygrywający.
	\item Dla $k \equiv 5 \pmod{7}$ zabranie $5$ kamieni doprowadza do stanu przegywającego, więc stan $k$ jest wygrywający. 
\end{itemize}
Powyższe rozpatrzenie przypadków kończy dowód indukcyjny.
Wystarczy teraz zauważyć, że $10000 \equiv 4 \pmod{7}$, więc to drugi gracz ma strategię wygrywającą.

\qed

\noindent
Analiza stanów jest znaną metodą analizy gier. Jest ona bardziej żmudna niż pomysłowa, toteż rzadko można ją spotkać na konkursach czy olimpiadach. Jednak pokazuje ona, jak wiele da się uzyskać z analizy ,,małych'' przypadków -- dzięki nim zarówno postawiliśmy słuszną hipotezę, jak i rozumowanie indukcyjne wypłynęło niejako naturalnie z bawienia się $k \in {1, 2, 3, 4, 5, 6, 7}$.


\vspace{10px}
\noindent
Niestety nie w każdym zadaniu analiza ,,małych'' przypadków coś daje. Czasem potrzebny jest pomysł, na który trzeba po prostu wpaść, często korzystając z metody prób i błędów. Jest ona tym skuteczniejsza, im więcej jest prób.

