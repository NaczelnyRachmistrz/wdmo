% Rozdział 10 – Reszty kwadratowe

\theory{Podzielności}

% wprowadzenie symbolu v_p

\heading{Wykładniki peadyczne}

\heading{Własności $v_p$}

% Source: XIII OMJ, finals

\heading{Przykład 1}

Niech $a$, $b$ będą nieparzystymi liczbami całkowitymi, dla których $a^bb^a$ jest kwadratem liczby całkowitej. Wykazać, że liczba $ab$ również jest kwadratem liczby całkowitej.

\heading{Rozwiązanie}


%Source: https://om.mimuw.edu.pl/static/app_main/problems/om66_1r.pdf P1
\heading{Przykład 2}

Dane są takie niezerowe liczby całkowite, że liczba
\[
	\frac{a}{b} + \frac{b}{c} + \frac{c}{a}
\]
jest całkowita. Wykazać, że iloczyn $abc$ jest sześcianem liczby całkowitej.

\heading{Rozwiązanie}

% Source: https://s3.amazonaws.com/aops-cdn.artofproblemsolving.com/resources/articles/olympiad-number-theory.pdf Ex 3.3.1

\heading{Przykład 3}

Wykazać, że dla żadnej dodatniej liczby całkowitej $n$ większej od $1$ liczba
\[
	1 + \frac{1}{2} + \frac{1}{3} + ... + \frac{1}{n}
\]
nie jest liczbą całkowitą.

\heading{Rozwiązanie}

\heading{Wzór Legendre'a}

\heading{Przykład 4}

Udowodnić, że dla wszystkich liczb całkowitych $n$ liczba 
\[
	\frac{1}{n+1}{{2n}\choose{n}}
\]
jest całkowita.

\heading{Rozwiązanie}
% wzór Langrange'a z silniami

