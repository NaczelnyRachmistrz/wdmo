%Source: 20^2 Problems in Number Theory P48
\begin{problem}{1}
	Wyznaczy wszystkie dodatnie liczby całkowite $n$, które są podzielne przez liczbę~$\left\lfloor \sqrt{n} \right\rfloor$.
\end{problem}


%Source: 20^2 Problems in Number Theory P154
\begin{problem}{2}
	Dane są dodatnie liczby całkowite $a$, $b$, $c$, że liczba $a^b$ dzieli $b^c$ oraz $a^c$ dzieli $c^b$. Wykazać, że $a^2$ dzieli $bc$.
\end{problem}

%Source: 20^2 Problems in Number Theory P168
\begin{problem}{3}
	Niech $1 = d_0 < d_1 < d_2 < ... < d_k = 4n$ będą kolejnymi dzielnikami liczby $4n$. Wykazać, że istnieje taka liczba $i$, dla której zachodzi równość $d_{i + 1} - d_i = 2$.
\end{problem}

%Source: 20^2 Problems in Number Theory P297
\begin{problem}{4}
	Znajdź wszystkie pary $(a, b)$ liczb całkowitych dodatnich, dla których obie liczby
	\[
		\frac{a^3b - 1}{a + 1} \quad \text{oraz} \quad \frac{b^3a - 1}{b - 1}
	\]
	są całkowite
\end{problem}

%Source: 20^2 Problems in Number Theory P170
\begin{problem}{5}
	Dane są parami różne dodatnie liczby całkowite $a$, $b$, $c$ oraz liczba pierwsza $p$, że liczby
	\[
		ab + 1, \; bc + 1, \; ca + 1
	\] 
	są podzielne przez $p$. Wykazać, że
	\[
		\frac{a + b + c}{3} \geqslant p + 2.
	\]
\end{problem}

\begin{problem}{6}
	Niech $a$ i $k$ będą pewnymi dodatnimi liczbami całkowitymi o tej własności, że dla dowolnej dodatniej liczby całkowitej $n$ istnieje liczba całkowita $b$, że
	\[
		n \mid a - b^k.
	\]
	Wykazać, że $a = c^k$ dla pewnej liczby całkowitej $c$.
\end{problem}

