\newpage
\solutions{Podzielności}

%Source: 20^2 Problems in Number Theory P48
\begin{problem}{1}
	Wyznaczy wszystkie dodatnie liczby całkowite $n$, które są podzielne przez liczbę~$\left\lfloor \sqrt{n} \right\rfloor$.
\end{problem}

\answer{Jedynymi liczbami spełniającymi warunki zadania są liczby postaci $k^2$, $k^2 + k$ oraz $k^2 + 2k$ dla pewnej liczby całkowitej $k$.}

\noindent
Rozpatrzmy taką dodatnią liczbę całkowitą $k$, że
\[
	(k + 1)^2 > n \geqslant k^2.
\]
Innymi słowy $k$ jest największą liczbą całkowitą, taką, że $k^2$ jest nie większe niż $n$. Wówczas
\[
	k + 1 > \sqrt{n} \geqslant k,
\]
czyli $\left\lfloor n \right\rfloor = k$. Jedynymi liczbami w przedziale $[k^2, (k + 1)^2)$, które są podzielne przez $k$ są $k^2$, $k^k + k$ i $k^2 + 2k$.


%Source: 20^2 Problems in Number Theory P154
\begin{problem}{2}
	Dane są dodatnie liczby całkowite $a$, $b$, $c$, że liczba $a^b$ dzieli $b^c$ oraz $a^c$ dzieli $c^b$. Wykazać, że $a^2$ dzieli $bc$.
\end{problem}

\noindent
Niech $p$ będzie dowolną liczbą pierwszą.
Z danych podzielności wynika, że
\[
	v_p(b^c) \geqslant v_p(a^b) \quad \text{oraz} \quad v_p(c^b) \geqslant v_p(a^c),
\]
czyli równoważnie
\[
	cv_p(b) \geqslant bv_p(a) \quad \text{oraz} \quad bv_p(c) \geqslant cv_p(a),
\]
\[
	cv_p(b) \geqslant bv_p(a) \quad \text{oraz} \quad bv_p(c) \geqslant cv_p(a).
\]
Mamy więc
\[
	v_p(bc) = v_p(b) + v_p(c) = \frac{b}{c}v_p(a) +  \frac{c}{b}v_p(a) \geqslant 2\sqrt{\frac{b}{c} \cdot \frac{c}{b} \cdot v_p(a)^2} = 2v_p(a),
\]
z czego wynika teza.


%Source: 20^2 Problems in Number Theory P168
\begin{problem}{3}
	Niech $1 = d_0 < d_1 < d_2 < ... < d_k = 4n$ będą kolejnymi dzielnikami liczby $4n$. Wykazać, że istnieje taka liczba $i$, dla której zachodzi równość $d_{i + 1} - d_i = 2$.
\end{problem}

\noindent
Załóżmy, że teza nie zachodzi -- wtedy musi zachodzić nastepujący fakt:
\begin{center} 
	Jeśli $d$ oraz $d + 2$ są dzielnikami liczby $4n$, to liczba $d + 1$ również. (*)
\end{center}
Wykażemy, że istnieje nieskończenie wiele takich liczb naturalnych $k$, że $2k$, $2k + 1$ oraz $2k + 2$ są dzielnikami liczby $n$. Na początku zauważmy, że $2$ i $4$ są dzielnikami $4n$. Z (*) liczba $3$ również nim będzie. 

\vspace{10px}

\noindent
Załóżmy więc, że dla pewnej liczby naturalnej $k$ liczby $2k$, $2k + 1$ oraz $2k + 2$ są dzielnikami liczby $4n$. Wówczas jedna z liczb $2k$, $2k + 2$ nie jest podzielna przez $4$. Bez straty ogólności przyjimy, że jest to $2k$. Skoro $2k$ nie dzieli się przez $4$ i jest dzielnikiem liczby $4n$, to liczba $4k$ również będzie dzielnikiem liczby $4n$. 

Analogicznie postępując z liczbą $2n + 1$, która jest nieparzysta, otrzymamy, że również liczba $4k + 2$ jest dzielnikiem liczby $4n$. Z (*) mamy, że $4k + 1$ musi być dzielnikiem $4n$. Otrzymaliśmy większą trójkę $(4k, 4k + 1, 4k + 2)$ kolejnych dzielników liczby $4n$.

\vspace{10px}

\noindent
Możemy w ten sposób uzyskać dowolnie duże trójki kolejnych dzielników $4n$, co jest oczywistą sprzecznością, gdyż dzielniki te są nie większe niż $4n$.

\vspace{10px}

%Source: 20^2 Problems in Number Theory P297
\begin{problem}{4}
	Znajdź wszystkie pary $(a, b)$ liczb całkowitych dodatnich, większych od 1, dla których obie liczby
	\[
		\frac{a^3b - 1}{a + 1} \quad \text{oraz} \quad \frac{b^3a - 1}{b - 1}
	\]
	są całkowite
\end{problem}

\answer{
	Szukanymi parami liczb są $a = 2$ i $b = 2$, $a = 1$ i $b = 3$, oraz $a = 3$ i $b = 3$.
}

\noindent
Zauważmy, że 
\[
	\frac{a^3b - 1}{a + 1} = a^2b - \frac{a^2b + 1}{a + 1} = a^2b - ab + \frac{ab - 1}{a + 1} = a^2b - ab + b - \frac{b + 1}{a + 1}.
\]
Wynika stąd, że liczba $b + 1$ jest podzielna przez liczbę $a + 1$. Mamy również
\[
	\frac{b^3a + 1}{b - 1} = b^2a + \frac{b^2a + 1}{b - 1} = b^2a + ab + \frac{ab + 1}{b - 1} = b^2a - ab + a + \frac{a + 1}{b - 1},
\]
czyli $a + 1$ jest podzielna przez liczbę $b - 1$. 

\vspace{10px}

\noindent
Liczba $a + 1$ jest dzielnikiem liczby $b + 1$ oraz wielokrotnością liczby $b - 1$. Stąd liczba $b + 1$ jest podzielna przez liczbę $b - 1$. Zauważmy, że jeśli $b \geqslant 4$, to
\[
	b + 1 > b - 1 > \frac{b + 1}{2},
\]
co przeczy wspomnianej podzielności.
Więc $b = 2$ lub $b = 3$. Jeśli $b = 2$, to $a + 1$ jest dzielnikiem $3$. Skoro jest większe od 1, to musi być równe 3, skąd $a = 2$. Dla $b = 3$ mamy, że $a + 1$ musi być dzielnikiem $4$ i musi być podzielne przez $2$. Stąd $a = 1$ lub $a = 3$. 

%Source: 20^2 Problems in Number Theory P170
\begin{problem}{5}
	Dane są parami różne dodatnie liczby całkowite $a$, $b$, $c$ oraz nieparzysta liczba pierwsza $p$, że liczby
	\[
		ab + 1, \; bc + 1, \; ca + 1
	\] 
	są podzielne przez $p$. Wykazać, że
	\[
		\frac{a + b + c}{3} \geqslant p + 2.
	\]
\end{problem}

\noindent
Zauważmy, że skoro $p \mid ab + 1$ oraz $p \mid bc + 1$, to
\[
	p \mid (ab + 1) - (bc + 1) = b(a - c).
\]
Skoro $p$ jest liczbą pierwszą, to dzieli ona $b$ lub dzieli ona $a - c$. Gdyby dzieliła liczbę $b$, to wówczas nie mogła by dzielić liczby $bc + 1 \equiv 1 \pmod{p}$. Stąd $p \mid a - c$, czyli 
\[
	a \equiv c \pmod{p}.
\] 
Rozumując analogicznie dla podzielności $p \mid bc + 1$ i $p \mid ca + 1$, otrzymamy $a \equiv b \pmod{p}$. Stąd
\[
	a \equiv b \equiv c \pmod{p}.
\]
Pozostaje zauważyć, że $a \not\equiv 1 \pmod{p}$, bo w przeciwnym wypadku 
\[
	ab + 1 \equiv a^2 + 1 \equiv 2 \pmod{p}.
\]
Przyjmijmy, że $2 \leqslant a < b < c$. Skoro dają one tę samą resztę z dzielenia przez $p$, to różnią się o wielokrotność liczby $p$, czyli co najmniej o $p$. Stąd
\[
	b \geqslant a + p \quad \text{oraz} \quad c \geqslant b + p \geqslant a + 2p,
\]
czyli
\[
	\frac{a + b + c}{3} \geqslant \frac{a + a + p + a + 2p}{3} = p + a \geqslant p + 2.
\]

\begin{problem}{6}
	Niech $a$ i $k$ będą pewnymi dodatnimi liczbami całkowitymi o tej własności, że dla dowolnej dodatniej liczby całkowitej $n$ istnieje liczba całkowita $b$, że
	\[
		n \mid a - b^k.
	\]
	Wykazać, że $a = c^k$ dla pewnej liczby całkowitej $c$.
\end{problem}

\noindent
\textit{Sposób 1}

\vspace{10px}

\noindent
Rozpatrzmy dowolną liczbę pierwszą $p$. Wykażemy, że $k \mid v_p(a)$, z czego wyniknie teza. Załóżmy nie wprost, że $v_p(a) = x \cdot k + r$, gdzie $x$ i $0 < r < k$ są dodatnimi liczbami całkowitymi. Weźmy $a = p^{xk}$. Wówczas istnieje takie $b$, że
\[
	p^{xk + k} \mid a - b^k.
\]
Zauważmy, że skoro $v_p(b^k)$ jest podzielne przez $k$, to może być albo nie mniejsze niż $xk + k$, albo nie większe niż $xk$. W obu przypadkach liczby $b^k$ i $a$ mają różne $v_p$, przy czym mniejsze z nich jest ściśle mniejsze niż $xk + k$. Na mocy Lematu 4 otrzymujemy 
\[
	v_p(a - b^k) = min(v_p(a), v_p(b^k)) < xk + k.
\]
Jest to sprzeczność z wyżej opisaną podzielnością.

\vspace{10px}

\noindent
\textit{Sposób 2}

\vspace{10px}

\noindent
Weźmy $n = a^2$. Wówczas na mocy założenia istnieją takie liczby całkowite $b$ oraz $d$, że
\[
	a - b^k = da^2,
\]
\[
	a(ad + 1) = b^k.
\]
Skoro $\mathrm{NWD}(a, ad + 1) = 1$, a iloczyn tych liczb jest $k$-tą potęgą liczby całkowitej, to każda z tych liczb jest $k$-tą potęgą liczby całkowitej.

\vspace{10px}

