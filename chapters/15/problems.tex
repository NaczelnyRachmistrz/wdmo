%Source: jakiś obóz informatyczny
\begin{problem}{1}
	Dany jest graf skierowany zawierający $2m$ krawędzi. Wykazać, że możemy usunąć pewne $m$ z nich, aby graf ten nie zawierał cyklu.
\end{problem}

%Source: https://mathsbeyondlimits.eu/wp-content/uploads/2021/04/MBL-brochure-2019.pdf
\begin{problem}{2}
	Wykazać, że w dla każdego turnieju da się ustawić graczy w rzędzie w pewnej kolejności, tak, aby każdy wygrał z graczem po swojej prawej stronie.
\end{problem}

%Source: Olympiad Combinatorics chapter 8 Ex 2
\begin{problem}{3}
	Niech $G$ będzie grafem skierowanym, w którym stopnie: wejściowy i wyjściowy każdego z wierzchołków są równe $2$. Wykazać, że możemy podzielić wierzchołki na trzy, być może puste, zbiory, aby żaden wierzchołek $v$ wraz z dwoma wierzchołkami, będącymi końcami wychodzącymi z $v$ krawędzi, nie znajdowały się wszystkie w jednym ze zbiorów.
\end{problem}

% QQ MBL
\begin{problem}{4}
	W pewnym turnieju brało udział $1000$ osób -- nazwijmy ich $A_1$, $A_2$, ..., $A_{1000}$. Parę uczestników tego turnieju $(A_i, A_j)$ nazwiemy \textit{zwycięską}, gdy nie istnieje inny uczestnik tego turnieju $A_k$, który pokonał obu uczestników z tej pary. Rozstrzygnąć, czy może się tak zdarzyć, aby każda z par $(A_1, A_2)$, $(A_2, A_3)$, ..., $(A_{999}, A_{1000})$, $(A_{1000}, A_{1})$ była zwycięska.
\end{problem}

%Source: https://math.stackexchange.com/questions/858110/strongly-regular-tournament
\begin{problem}{5}
	Dany jest turniej o $n$ uczestnikach oraz pewna dodatnia liczba całkowita $k$. Każdy z uczestników uzyskał jednakową liczbę zwycięstw oraz dla dowolnych dwóch różnych uczestników tego turnieju $a$, $b$, liczba uczestników, którzy przegrali zarówno z $a$, jak i z $b$ wynosi $k$. Wykazać, że liczba $n + 1$ jest podzielna przez 4.
\end{problem}

%Source: https://om.mimuw.edu.pl/static/app_main/camps/oboz2016.pdf Pierwszy mecz P5
\begin{problem}{6}
	Wykazać, że wierzchołki grafu można pokolorować na jeden z $k$ kolorów, tak, by żadne dwa wierzchołki tego samego koloru nie były połączone krawędzią wtedy i tylko wtedy, gdy krawędzie tego grafu da się tak skierować, by nie istniała ścieżka składająca się z $k + 1$ różnych wierzchołków.
\end{problem}