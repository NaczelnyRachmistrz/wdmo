\hints{Wielomiany 2}

\begin{hints_list}
	\item Skorzystaj z Twierdzenia $1$.
	\item Weź taki wielomian, aby $W(n)$, $W(W(n))$, ..., dawały reszty $1$ z dzielenia przez $n$.
	\item Rozpatrz wielomian $W(x) - W(-x)$.
	\item Wykaż, że $P(x) > Q(x)$ dla wszystkich liczb rzeczywistych $x$ lub $P(x) < Q(x)$ dla wszystkich liczb rzeczywistych $x$.
	\item Niech $q$ będzie liczbą wymierną. Zauważ, że wielomian $W(x) + W(2q - x)$ przyjmuje tylko wartości wymierne.
	\item Małgosia jest w stanie wygrać w $2$ ruchach.
\end{hints_list}