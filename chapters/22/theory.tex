% Rozdział 22 - Wielomiany 2

\theory{Wielomiany 2}

\heading{Twierdzenie 1}

\noindent
Dany jest wielomian $P(x)$ o współczynnikach całkowitych oraz dwie różne liczby całkowite $a$, $b$. Wówczas
\[
	a - b \mid P(a) - P(b).
\]

\heading{Dowód}

% tutaj trochę opisać, że a = b mod n, to f(a) = f(b)


\heading{Twierdzenie 2}

\noindent
Dane są liczby całkowite $a_0$, $a_1$, ..., $a_n$, liczba wymierna $\frac{p}{q}$, $NWD(p, q) = 1$, oraz wielomian 
\[
	W(x) = a_nx^n + a_{n - 1}x^{n - 1} + ... + a_1x + a_0, \; \text{ przy czym } \; W\left(\frac{p}{q}\right) = 0.
\] 
 Wtedy
\[
	p \mid a_o \quad \text{oraz} \quad q \mid a_{n}.
\]

\heading{Dowód}

\heading{Przykład 1}

Dany jest niestały wielomian $W(x)$ o wspołczynnikach całkowitych. Wykazać, że istnieje nieskończenie wiele liczb pierwszych $p_i$, dla których istnieje taka liczba całkowita $n$, że $p_i \mid W(n)$.

\heading{Dowód}

\heading{Twierdzenie 3}

\noindent
Niech $W(x)$ będzie wielomianem o współczynnikach rzeczywistych. Jeśli $a < b$ są liczbami rzeczywistymi, przy czym $W(a) < 0$ i $W(b) > 0$, to istnieje takie $x \in (x, y)$, że $W(x) = 0$

% napisać, że to jest własność Darboux i jakiś rysunek dać

% Source: https://om.mimuw.edu.pl/static/app_main/camps/oboz2018.pdf P3
\heading{Przykład 2}

\noindent
Dane są parami różne liczby całkowite $a_1, a_2, ..., a_n$, gdzie $n \geqslant 1$. Wykazać, że wielomian 
\[
	P(x)=((x − a_1)(x − a_2)\cdot ...\cdot (x − a_n))^2 + 1
\]
nie jest iloczynem dwóch wielomianów dodatnich stopni o współczynnikach całkowitych.

\heading{Dowód}