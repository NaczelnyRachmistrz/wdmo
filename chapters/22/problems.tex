\begin{problem}{1}
	Niech $a,b,c$ będą trzema parami różnymi liczbami całkowitymi. Wykazać, że nie istnieje taki wielomian $P(x)$ o współczynnikach całkowitych, że zachodzą równości
    \[
        P(a) = b,\; P(b) = c,\; P(c) = a.
    \]
\end{problem}

\begin{problem}{2}
	Udowodnić, że istnieje taki wielomian $W$ stopnia $100$ o współczynnikach całkowitych, że dla każdej liczby całkowitej $n$ liczby
	\[
		n,\;\; W(n),\;\; W(W(n)),\;\; W(W(W(n))), ...
	\]
	są parami względnie pierwsze.
\end{problem}

\begin{problem}{3}
	Dany jest wielomian $W(x)$ o współczynnikach rzeczywistych, który ma niezerowy współczynnik przy $x^1$. Wykazać, że istnieje taka liczba rzeczywista $a$, że
	\[
		W(a) \neq W(-a).
	\]
\end{problem}

\begin{problem}{4}
	Niech $P$, $Q$ będą wielomianami o współczynnikach rzeczywistych, dla których zachodzi równość 
	\[
		P(Q(x)) = Q(P(x)).
	\] 
	Wykazać, że jeśli nie istnieje taka liczba rzeczywista $x$ dla której $P(x) = Q(x)$, wówczas również nie istnieje taka liczba rzeczywista $x$, dla której prawdą jest, że $P(P(x)) = Q(Q(x))$.
\end{problem}

\begin{problem}{5}
	Znaleźć wszystkie wielomiany $W$ o współczynnikach rzeczywistych, mających następującą własność:
	jeśli $x + y$ jest liczbą wymierną, to $W(x) + W(y)$ jest liczbą wymierną.
\end{problem}

\begin{problem}{6}
	Jaś i Małgosia grają w grę. Jaś ma pewną funkcję 
	\[
		f(x)=a_n x^n + \dots + a_1x+a_0,
	\] 
	gdzie $a_0$, $\dots$, $a_n$ to są liczby dodatnie całkowite i $n\geqslant 3$. W każdym ruchu Małgosia podaje pewną liczbę rzeczywistą $x$, po czym Jaś musi podać Małgosi wartość~$f(x).$ Taki ruch powtarzają kilkukrotnie. Małgosia wygrywa, jak wie, ile wynoszą wszystkie liczby $a_0$, $a_1$, $\dots$, $a_n.$ Znajdź, w zależności od $n$, taką liczbę ruchów, że Małgosia na pewno w tylu ruchach wygra.
\end{problem}

