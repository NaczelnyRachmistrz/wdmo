% Rozdział 20 - Klasyczne zadania z teorii liczb

\theory{Klasyczne zadania z teorii liczb}

\heading{Przykład 1}

\noindent
Rozstrzygnąć, czy istnieje taka liczba naturalna $n$, że sumy cyfr liczb $2^n$ oraz $2^{n + 1}$ są sobie równe.

\heading{Rozwiązanie}


% Source: https://omj.edu.pl/uploads/attachments/broszura_OM_2014.pdf

\heading{Przykład 2}

\noindent
Udowodnij, że w rozwinięciu dziesiętnym liczby $(5 + \sqrt{26})^{2000}$ na pierwszych $2000$ miejscach nie występuje liczba 7.

\heading{Rozwiązanie}

% Source: Burek P67
\heading{Przykład 3}

\noindent
Dodatnie liczby całkowite $a$, $b$, $c$ są względnie pierwsze oraz spełniają równość
\[
	\frac{1}{a} + \frac{1}{b} = \frac{1}{c}.
\]
Wykazać, że liczba $a + b$ jest kwadratem liczby całkowitej.

\heading{Rozwiązanie}

% Source: QQ MBL
\heading{Przykład 4}

\noindent
Ułamek nieskracalny $\frac{a}{b}$ nazwiemy \textit{zbalansowanym}, jeśli liczby $a$ oraz $b$ są iloczynem tej samej liczby liczb pierwszych -- niekoniecznie różnych. Wykazać, że dla dowolnej dodatniej liczby całkowitej $k$, istnieją takie dodatnie liczby całkowite $x$, $y$, że każda z liczb
\[
	\frac{x}{y}, \; \frac{x + 1}{y + 1}, \; \frac{x + 2}{y + 2}, \; ..., \; \frac{x + k}{y + k}
\]  
jest zbalansowana.


\heading{Rozwiązanie}

%enumeratywn


