% Rozdział 20 - Klasyczne zadania z teorii liczb

\theory{Klasyczne zadania z teorii liczb}

\noindent
W niniejszym rozdziale pochylimy się nad kilkoma motywami, które można nazwać ,,klasycznymi''. Najpierw przyjrzyjmy się znanemu trikowi, który często jest używamy w zadaniach, w których mowa o sumie cyfr.

\vspace{10px}

\heading{Przykład 1}

\noindent
Niech $S(n)$ oznacza sumę cyfr liczby $n$ w zapisie dziesiętnym.
Rozstrzygnąć, czy istnieje taka liczba naturalna $n$, że 
\[
	S(2^n) = S(2^{n + 1}).
\]

\vspace{5px}

\heading{Rozwiązanie}

\noindent
Wykażemy, że dla każdej liczby całkowitej $n$ zachodzi przystawanie
\[
	S(n) \equiv n \pmod{9}.
\]
Niech
\[
	n = a_k \cdot 10^k + a_{k - 1} \cdot 10^{k - 1} + ... + 10 \cdot a_1 + a_0,
\]
gdzie liczby $a_i$ są cyframi liczby $n$. Zauważmy, że
\begin{gather*}
	10^t \equiv 1^t \equiv 1 \pmod{9} \\
	n = a_k \cdot 10^k + a_{k - 1} \cdot 10^{k - 1} + ... + 10 \cdot a_1 + a_0 \equiv \\
	\equiv a_k + a_{k - 1} + ... + a_1 + a_0 = S(n) \pmod{9}.
\end{gather*}

\vspace{10px}
\noindent
Mamy więc
\begin{align*}
	2^{n + 1} \equiv S(2^{n + 1}) = S(2^n)  \equiv 2^n \pmod{9}\\
	2^{n + 1} - 2^n = 2^n \equiv 0 \pmod{9}.
\end{align*}
Stąd liczba $2^n$ musiałaby być podzielna przez $9$, co daje sprzeczność.

\qed


\vspace{10px}

\heading{Przykład 2}

\noindent
Dodatnie liczby całkowite $a$ i $b$ spełniają równość
\[
	3a^2 + a = 2b^2 + b.
\]
Wykazać, że liczba $b - a$ jest kwadratem liczby całkowitej.

\vspace{5px}

\heading{Rozwiązanie}

\noindent
Daną równość można przekształcić do postaci
\[
	a^2 = 2b^2 - 2a^2 + (b - a) = (b - a)(2a + 2b + 1).
\]
Wykażemy, że
\[
	\mathrm{NWD}(b - a, 2a + 2b + 1) = 1.
\]
Załóżmy, że pewna liczba pierwsza $p$ dzieli zarówno $b - a$, jak i $2a + 2b + 1$. Wówczas w szczególności dzieli ich iloczyn, czyli liczbę $a^2$. Stąd, skoro $p$ jest liczbą pierwszą, to $p \mid a$. Rozumując dalej mamy
\[
	 p | b - a \;\; \text{i} \;\; p | a \implies \;\; p | b \implies p|2a + 2b
\]
Stąd $p$ nie może dzielić liczby $2a + 2b + 1$, co daje sprzeczność.

\vspace{10px}
\noindent
Zauważmy, że skoro $b - a$ i $2a + 2b + 1$ są dodatnie i względnie pierwsze oraz ich iloczyn jest kwadratem liczby całkowitej, to obie muszą być kwadratami liczb całkowitych. Istotnie, rozpatrzmy dowolną liczbę pierwszą $p$ -- może on dzielić co najwyżej jedną z liczb $b - a$ i $2a + 2b + 1$. Skoro wykładnik $p$ w rozkładzie $a^2$ na czynniki pierwsze jest parzysty, to mamy że dla jednej z liczb $b - a$ i $2a + 2b + 1$ jest on parzysty, a dla drugiej równy zeru, czyli także parzysty. Dowodzi to postulowanej własności.

\qed

\noindent
W powyższym zadaniu kluczowa była obserwacja, że pewne dwie liczby są względnie pierwsze. Warto zwracać na takie rzeczy uwagę przy podobnych zadaniach z teorii liczb. Teraz przejdziemy do zadania z bardzo ciekawym motywem.

\vspace{10px}

% Source: QQ MBL
\heading{Przykład 3}

\noindent
Ułamek $\frac{a}{b}$ nazwiemy \textit{zbilansowanym}, jeśli liczby $a$ oraz $b$ są iloczynem tej samej liczby liczb pierwszych -- niekoniecznie różnych. Dla przykładu ułamki $\frac{2 \cdot 3}{5 \cdot 7}$ oraz $\frac{3 \cdot 11}{5^2}$ są zbilansowane, a ułamek $\frac{3}{2 \cdot 5}$ nie. Wykazać, że dla dowolnej dodatniej liczby całkowitej~$k$, istnieją takie dodatnie liczby całkowite $x$, $y$, że każda z liczb
\[
	\frac{x}{y}, \; \frac{x + 1}{y + 1}, \; \frac{x + 2}{y + 2}, \; ..., \; \frac{x + k}{y + k}
\]  
jest zbilansowana.

\vspace{5px}

\heading{Rozwiązanie}

\noindent
Niech $\Omega(n)$ oznacza liczbę niekoniecznie różnych dzielników pierwszych liczby $n$.
Zdefiniujmy dla każdej liczby $n$ jako $f(n)$ ciąg 
\[
	f(n) = (\Omega(n),\;\; \Omega(n + 1),\;\; ...,\;\; \Omega(n + k)).
\]
Zauważmy, że jeśli $f(x) = f(y)$, to ułamki
\[
	\frac{x}{y}, \; \frac{x + 1}{y + 1}, \; \frac{x + 2}{y + 2}, \; ..., \; \frac{x + k}{y + k}
\]
spełniają warunki zadania, bo dla każdego z rozpatrywanych $i$ mamy
\[
	\Omega(x + i) = \Omega(y + i) \implies \frac{x + i}{y + i} \;\; \text{jest zbilansowany}.
\]
Wykażemy, że takie $x$ i $y$ istnieją za pomocą argumentu ilościowego. Rozpatrzmy liczby od $1$ do $2^n$ dla pewnego $n$. Jeśli $a$ jest jedną z nich, to może mieć ona co najwyżej $n$ dzielników pierwszych. W przeciwnym przypadku byłaby równa co najmniej niż $2^{n + 1}$. 

\noindent
Rozpatrzmy teraz liczby z przedziału od $1$ do $2^n - k$ dla pewnego $n$. Jak wykazaliśmy wyżej, jeśli $1 \leqslant a \leqslant 2^n - k$, to $f(a)$ zawiera $k$ liczb z przedziału od $0$ do $n$, stąd $f$ może przyjąć jedną z $k(n + 1)$ wartości. Jeśli wszystkie 
\[
	f(1), \;\; f(2), \;\; ..., \;\; f(2^n - k)
\]
miałyby być parami różne, to ich liczba musiałaby być nie większa niż liczba możliwych wartości $f$ na tym przedziale, skąd
\begin{align*}
	2^n - k &\leqslant k(n + 1), \\
	2^n &\leqslant k(n + 2).
\end{align*}
Lewa strona tej równości rośnie wykładniczo, a prawa liniowo, stąd dla odpowiednio dużej liczby $n$ taka równość nie będzie zachodzić. Ta sprzeczność dowodzi, że istnieją pewne $x$ i $y$, że $f(x) = f(y)$, z czego jak już pokazaliśmy wynika teza.

\qed

\noindent
Powyższy motyw jest dość ciekawy. Nie konstruujemy $x$ i $y$ bezpośrednio -- w tym przypadku tego raczej nie da się zrobić. Zamiast tego używamy argumentu, który opiera się na tym, że możliwości wyboru $x$ i $y$ jest dużo w porównaniu do możliwości wyboru konfiguracji liczb dzielników pierwszych liczb $x + i$ oraz $y + i$. Jeśli chcemy wykazać istnienie czegoś, to czasami przydaje się podobne myślenie -- argumenty tego typu rozwiązywały już zadania na finale OM.
\vspace{10px} 
