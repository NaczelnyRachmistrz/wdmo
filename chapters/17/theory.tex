% Rozdział 17 – Rozstrzygnij, czy

\theory{Rozstrzygnij, czy}


% IMO 2010 P5
\heading{Przykład 1}

\noindent
Każde z sześciu pudeł $B_1$, $B_2$, $B_3$, $B_4$, $B_5$, $B_6$ na początku zawiera po jednej monecie. Następujące operacje są dozwolone

\begin{enumerate}
	\item Wybrać niepuste pudełko $B_j$, $1\leq j \leq 5$, usunąć jedną monetę z $B_j$ i dołożyć dwie monety do $B_{j+1}$;

	\item  Wybrac niepuste pudełko $B_k$, $1\leq k \leq 4$, usunąć jedną monetę z $B_k$ i zamienić kolejnośc pudełek $B_{k+1}$ i $B_{k+2}$.

\end{enumerate}

\noindent
Rozstrzygnąc, czy instieje taki ciąg operacji, że każde z pudełek $B_1$, $B_2$, $B_3$, $B_4$, $B_5$ stanie się puste, zaś pudełko $B_6$ będzie zawierało $2000^{2000^{2000}}$ monet.

\heading{Rozwiązanie}

% trzeba się pobawić i zobaczyć, jakie liczby da się uzyskać

%Source: Burek
\heading{Przykład 2}

\noindent
Dane są takie liczby całkowite $a$, $b$, $c$, $d$, że nie istnieją takie niezerowe liczby całkowite $x$, $y$, $z$, $t$, że
\[
	ax^2 + by^2 + cz^2 + dt^2 = 0.
\]
Rozstrzygnąć, czy wynika z tego, że liczby $a$, $b$, $c$, $d$ mają ten sam znak.

\heading{Rozwiązanie}

%tutaj ciężej jest wpaść na to, analizująć drobne przypadki
% trzeba zauważyć, że trzbe skorzystać z całkowitości – stąd teoria liczb

% metoda nieskończonego schodzenia