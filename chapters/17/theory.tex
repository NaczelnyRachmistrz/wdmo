% Rozdział 17 – Rozstrzygnij, czy

\theory{Rozstrzygnij, czy}

\noindent
Pochylimy się nad zadaniami, w którym nie wiemy, czy mamy dowodzić tezy, czy też szukać do niej kontrprzykładu. Czasami, mimo że w zadaniu napisano ,,rozstrzygnij'', tak właściwie jest jasne, że tezę trzeba udowodnić. Nie będziemy w tym rozdziale zajmować się tego typu rzeczami -- w poniższych zadaniach rozpoznanie w którym kierunku należy prowadzić rozumowanie, jest co najmniej nietrywialne.

\vspace{10px}
\noindent
Poniższy przykład to nieco uproszczone zadanie 5 z Międzynarodowej Olimpiady Matematycznej z $2010$ roku. Sprawiło ono kłopot wielu bardzo dobrym zawodnikom -- rozwiązał je jeden z sześciu reprezentantów Polski.

% IMO 2010 P5
\heading{Przykład 1}

\noindent
Każde z sześciu pudeł $B_1$, $B_2$, $B_3$, $B_4$, $B_5$, $B_6$ na początku zawiera po jednej monecie. Następujące operacje są dozwolone

\begin{enumerate}
	\item Wybrać niepuste pudełko $B_j$, $1\leq j \leq 5$, usunąć jedną monetę z $B_j$ i dołożyć dwie monety do $B_{j + 1}$;

	\item  Wybrać niepuste pudełko $B_k$, $1\leq k \leq 4$, usunąć jedną monetę z $B_k$ i zamienić kolejność pudełek $B_{k+1}$ i $B_{k+2}$.

\end{enumerate}

\noindent
Rozstrzygnąć, czy istnieje taki ciąg operacji, że każde z pudełek $B_1$, $B_2$, $B_3$, $B_4$, $B_5$ stanie się puste, zaś pudełko $B_6$ będzie zawierało $2000^{2000}$ monet.

\vspace{5px}

\heading{Rozwiązanie}

\noindent
Wykażemy, że istnieje ciąg ruchów spełniający warunki zadania. Przyjmiemy, że 
\[
	(a_1, a_2, ....) \rightarrow (b_1, b_2, ...)
\] 
jeśli można wykonać pewną liczbę ruchów, że z ciągu kolejnych pudełek w których jest $a_1, a_2, ....$ monet otrzymamy ciąg $b_1, b_2, ....$ monet.

\vspace{10px}

\noindent
\underline{Obserwacja.} $(a, 0, 0) \rightarrow (0, 2^a, 0)$ dla każdego $a \geqslant 1.$

\vspace{5px}
\noindent
Zauważmy,~że
\[
	(a, 0, 0) \rightarrow (a - 1, 2, 0)
\]
oraz
\[
	(x, 2^y, 0) \rightarrow (x, 2^y - 1, 2) \rightarrow ... \rightarrow (x, 0, 2^{y + 1}) \rightarrow (x - 1, 2^{y}, 0).
\]
Wykonując najpierw pierwsze~z powyższych przekształceń,~a następnie wykonując ${a - 1}$ razy drugie~z nich otrzymamy tezę.

\vspace{10px}
\noindent
Najpierw wykonany takie ruchy, aby otrzymać w ostatnim pudełku bardzo dużą liczbę, a następnie cofniemy ją prawie na początek.
\begin{align*}
	&(1, 1, 1, 1, 1, 1) \rightarrow (0, 3, 1, 1, 1, 1) \rightarrow (0, 1, 5, 1, 1, 1)  \rightarrow \\
	\rightarrow &(0, 1, 1, 9, 1, 1)  \rightarrow (0, 1, 1, 1, 17, 1)  \rightarrow (0, 1, 1, 1, 0, 35) \rightarrow \\
	\rightarrow  &(0, 1, 1, 0, 35, 0)  \rightarrow (0, 1, 0, 35, 0, 0)  \rightarrow (0, 0, 35, 0, 0, 0).
\end{align*}

\noindent
Na mocy obserwacji wykonujemy ruchy
\[
	 (0, 0, 35, 0, 0, 0) \rightarrow  (0, 0, 1, 2^{34}, 0, 0) \rightarrow  (0, 0, 1, 0, 2^{2^{34}}, 0) \rightarrow  (0, 0, 0, 2^{2^{34}}, 0 , 0).
\]
Zauważmy, że 
\[
	2^{2^{34}} > 2^{1000^3} > \left(2^{2000}\right)^{11} > 2000^{2000}.
\] 
Przyjmijmy $A = 2000^{2000}$. Wykonując drugi ruch dla $k = 4$ tak właściwie jedynie zabieramy monetę z czwartego pudełka, gdyż pudełka: piąte i szóste są puste. Możemy więc zabrać ich tyle, by w czwartym pudełku zostało $\frac{1}{4}A$ monet. Wówczas wykonując ruchy pierwszego rodzaju otrzymujemy
\[
	(0, 0, 0, \tfrac{1}{4}A, 0, 0) \rightarrow (0, 0, 0, 0, \tfrac{1}{2}A, 0) \rightarrow (0, 0, 0, 0, 0, A).
\]

\qed

\noindent
Kluczowym elementem rozwiązywania tego typu zadań jest pobawienie się danymi operacjami~i zobaczenie jakie liczby da się uzyskać. Jeśli zrobimy to dobrze, to unikniemy sytuacji, w której przez dłuższy czas bezskutecznie próbujemy dowodzić, że odpowiedzią na postawione pytanie jest ,,Nie''.

\vspace{5px}

% trzeba się pobawić i zobaczyć, jakie liczby da się uzyskać

%Source: 20^2 Problems in Number Theory Dominika Burka
\heading{Przykład 2}

\noindent
Dane są takie liczby całkowite $a$, $b$, $c$, $d$, że nie istnieją takie niezerowe liczby całkowite~$x$, $y$, $z$, $t$, że
\[
	ax^2 + by^2 + cz^2 + dt^2 = 0.
\]
Rozstrzygnąć, czy wynika z tego, że liczby $a$, $b$, $c$, $d$ mają ten sam znak.

\vspace{5px}

\heading{Rozwiązanie}

\noindent
Wykażemy, że postulowana implikacja nie zachodzi. Rozpatrzmy liczby
\[
	a = b = 3, \; c = d = -1.
\]
Udowodnimy, że jeśli zachodzi równość
\[
	3x^2 + 3y^2 = z^2 + t^2,
\]
to wówczas $x = y = z = t = 0$. Rozumując nie wprost, załóżmy, że istnieje niezerowe rozwiązanie powyższego równania. Rozpatrzmy takie rozwiązanie, dla którego suma~${|x| + |y| + |z| + |t|}$ jest najmniejsza możliwa. Wiemy, że
\[
	z^2 + t^2 = 3x^2 + 3y^2 \equiv 0 \pmod{3}
\]
Znanym faktem jest, że dla dowolnej liczby całkowitej $n$ mamy
\[
	n^2 \equiv 0 \pmod{3},\; \text{gdy } n \equiv 0 \pmod{3} \quad \text{oraz} \quad n^2 \equiv 1 \pmod{3},\; \text{gdy } n \equiv 1, 2 \pmod{3}.
\]
Rozpatrując wszystkie możliwe przypadki, w zależności od reszt z dzielenia przez $3$ jakie dają liczby $z$ i $t$, łatwo otrzymać, że
\[
	z^2 + t^2  \equiv 0 \pmod{3} \iff z \equiv t \equiv 0 \pmod{3}.
\]
Mamy więc $z = 3z_0$ oraz $t = 3t_0$ dla pewnych liczb całkowitych $z_0$,\; $t_0$. Wówczas
\begin{align*}
	3x^2 + 3y^2 &= z^2 + t^2 = 9z_0^2 + 9t_0^2, \\
	3z_0^2 + 3t_0^2 &= x^2 + y^2.
\end{align*}
Stąd czwórka $(z_0, t_0, x, y)$ również jest niezerowym rozwiązaniem danego równania. Jeśli obie z liczb $z_0$, $t_0$ są równe zeru, to łatwo zauważyć, że pozostałe dwie również, co przeczy niezerowości danego rozwiązania. W przeciwnym wypadku
\[
	|z_0| + |t_0| + |x| + |y| < |x| + |y| + |z| + |t|
\]
co przeczy minimalności $|x| + |y| + |z| + |t|$.

\qed

\noindent
Metodę zastosowaną powyżej nazywa się \textit{nieskończonym schodzeniem}. Z pewnego rozwiązania równania uzyskujemy rozwiązanie mniejsze. Nie można jednak schodzić w nieskończoność. Skoro działamy w liczbach całkowitych, to wartość wyrażenia ${|x| + |y| + |z| + |t|}$ nie może przyjmować dowolnie małych wartości.

\vspace{10px}
\noindent
Ciężko w powyższym zadaniu wpaść na rozwiązanie obserwując jak teza zachowuje się dla małych liczb $a$, $b$, $c$, $d$. Nie w każdym zadaniu analiza szczególnych przypadków jest pomocna. Ważna była obserwacja, że na mocy założeń liczby $x$, $y$, $z$ i $t$ są całkowite. Trzeba jakoś skorzystać z tej własności -- tutaj nasuwa się modulo.

\vspace{10px}