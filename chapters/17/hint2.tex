\hints{Rozstrzygnij, czy}

\begin{hints_list}
	\item Skorzystaj z przystawania modulo $3$.
	\item Zauważ, że $(-x)^2 = x^2$.
	\item Wykaż, że $f(a^k) = kf(a)$.
	\item Rozpatrz konfigurację, w której jest dużo trójkątów zegarowych oraz taką, w której jest ich mało. Płynnie przesuwaj punkty tak, aby z jednej otrzymać drugą. W pewnym momencie musisz otrzymać szukaną konfigurację.
	\item Wykaż, że zawsze da się, w pewnej liczbie ruchów przeskoczyć z liczby $n$ na liczbę $n + 1$.
	\item Zdefiniuj proste jako $y = a_i \cdot x + b_i$ dla pewnych liczb $a_i$ oraz $b_i.$
\end{hints_list}