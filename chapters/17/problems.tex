
% Source: https://skm.katowice.pl/pliki/szkice/rejonowe/rozw_rejon_2019.pdf
\begin{problem}{1}
	Rozstrzygnąć, czy istnieją liczby całkowite $x$, $y$, spełniające równanie
	\[
		x^2 - 999y^2 = 1001.
	\]
\end{problem}

% Source: https://dominik-burek.u.matinf.uj.edu.pl/Rabka2017.pdf
\begin{problem}{2}
	Niech $S(\mathcal{X})$ oznacza sumę elementów zbioru $\mathcal{X}$, zaś $S_2(\mathcal{X})$ sumę kwadratów elementów zbioru $\mathcal{X}$. Rozstrzygnąć, czy istnieją takie rozłączne zbiory liczb całkowitych $A$ oraz $B$, które zawierają po $2020$ elementów, że
	\[
		S(A) = S(B) \quad \text{oraz} \quad S_2(A) = S_2(B).
	\]
\end{problem}


%Source: https://om.mimuw.edu.pl/static/app_main/camps/oboz2019.pdf
\begin{problem}{3}
	Rozstrzygnąć, czy istnieje taka różnowartościowa funkcja $f$ określona na zbiorze liczb całkowitych dodatnich i przyjmująca wartości całkowite nieujemne, że
	\[
		f(mn) = f(m) + f(n)
	\]
	dla dowolnych liczb całkowitych $m$, $n$.
\end{problem}


%Source: https://artofproblemsolving.com/community/c6h1513472p8997062 4 IGO P4 Junior
\begin{problem}{4}
	$P_1,P_2,\ldots,P_{100}$ są $100$ punktami na płaszczyźnie, z których żadne 3 nie leżą na jednej prostej. Dla każdych trzech punktów, nazwijmy trójkąt przez nie tworzony \textit{zegarowym}, jeśli numery wierzchołków są uporządkowane rosnąco zgodnie z ruchem wskazówek zegara. Czy liczba trójkątów zegarowych może wynosić dokładnie $1111$?
\end{problem}

%Source: https://artofproblemsolving.com/community/c6h1172747p5641811
\begin{problem}{5}
	Nieśmiertelny konik polny skacze po osi liczbowej, zaczynając od $0$. Długość $k$-tego skoku wynosi $2^k + 1$. Konik sam decyduje, czy skoczy w prawo czy w lewo. Czy jest możliwe, aby każda liczba całkowita została w pewnym momencie odwiedzona przez konika polnego? Konik polny może odwiedzić pewną liczbę więcej niż raz.
\end{problem}

%Source: od Szymczyka
\begin{problem}{6}
	Rozstrzygnąć, czy istnieje nieskończony zbiór prostych na płaszczyźnie, taki, że dowolne dwie proste do niego należące przecinają się w pewnym punkcie o współrzędnych $(x, y)$, gdzie $x$ i $y$ są liczbami całkowitymi.
\end{problem}


