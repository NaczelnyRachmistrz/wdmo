\thispagestyle{empty}\addtocounter{page}{-1}
%\vspace*{\fill}
	\begin{center}
		\fontsize{15}{15}\selectfont
		Paweł Gadziński

		\vspace{50px}

		\fontsize{40}{40}\selectfont
		\textcolor{kolor}{Wstęp do matematyki olimpijskiej}

		\vspace{20px}

		\fontsize{25}{25}\selectfont
		\textcolor{kolor}{zadania niegeometryczne}

		\vspace{40px}

		\fontsize{15}{15}\selectfont
		wersja z 7 stycznia 2022
	\end{center}
\vspace*{\fill}
\begin{center}
		\fontsize{15}{15}\selectfont
		Książka w wersji online dostępna pod adresem: \\
		\url{http://wdmo.pl}
\end{center}
\newpage
	\vspace*{\fill}
	\begin{center}
		\addcontentsline{toc}{section}{Wstęp}
		\fontsize{20}{20}\selectfont
		\textbf{Wstęp}
		\vspace{30px}
	\end{center}
		\noindent
		Książka ta jest skierowana do osób, które mają już pewne doświadczenie z matematyką olimpijską i chcieliby powalczyć o tytuł finalisty Olimpiady Matematycznej. Jeśli czytelniczka/czytelnik nie ma takowego doświadczenia, to zachęcamy, aby najpierw zapoznać się z zadaniami i materiałami z Olimpiady Matematycznej Juniorów.

		\vspace{10px}
		\noindent
		Każdy z rodziałów zaczyna się omówieniem pewnego zagadnienia teoretycznego. Postaram się zamieścić w tej książce wszystkie niezbędne narzędzia, które moga przydać się na drugim etapie OM. 

		\vspace{10px}
		\noindent
		Jednak to nie wiedza teoretyczna jest najważniejsza na Olimpiadzie. Znacznie istotniejsze jest jej kreatywne zastosowanie, często wykraczające poza schematy. Dlatego częścią każdego rozdziału jest kilka zadań. Są one dobrane tak, aby były niesztampowe oraz poszerzały horyzonty myślowe. Absolutnie nie należy traktować ich jako ćwiczeń do poprzedzającej ich części teoretycznej. Mogą one do niej nawiązywać, ale nie zawsze będą. Trenowanie na zadaniach, które są bardziej zróżnicowane pod względem tematyki jest bardziej efektywne, czego dowodzą liczne badania prowadzone w tym zakresie(Roher, Dedrick i Burgess, 2014). 

		\vspace{10px}
		\noindent
		Do każdego z zadań zamieszczono do trzech podpowiedzi. Zachęcamy do skorzystania z nich przed zobaczeniem na rozwiązanie. Przy większości zadań przeczytanie wszytskich podpowiedzi powinno pozwolić na samodzielne wymyślenie rozwiązania, a to daje satysfakcję.

		\vspace{10px}
		\noindent
		Książka jest dostępna za darmo, ale jednak proszę osoby, które z niej korzystają o drobną przysługę. Jeśli zostanie zauważony jakiś błąd, nieścisłość, czy też uważasz, że można coś napisać lepiej, to bardzo prosiłbym o zwrócenie uwagi przez formularz dostępny w wersji on-line publikacji lub też pisanie na maila \textit{admin@wdmo.pl}.

		\vspace{10px}
 
		\hspace*{\fill}Paweł Gadziński

\vspace*{\fill}
\newpage